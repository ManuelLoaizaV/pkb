\subsection{Funciones Anal\'iticas}

Sea $U \subset \BC$ abierto. Decimos que $f: U \to \BC$ es \textbf{anal\'itica} si para todo $z_0 \in U$
existe un abierto $U_{z_0}$ tal que $z_0 \in U_{z_0} \subset U$ y una serie de potencias formal
$\sum_{k = 0}^{\infty} a_k z^k$ tal que
\[
  f(z) = \sum_{k = 0}^{\infty} a_k (z - z_0)^k
\]
para todo $z \in U_{z_0}$.

\begin{theorem}
  Toda funci\'on anal\'itica es holomorfa.
\end{theorem}

\begin{theorem}
  La derivada de una funci\'on anal\'itica es tambi\'en anal\'itica.
\end{theorem}

\begin{remark}
  Toda funci\'on anal\'itica es real anal\'itica pero la inversa no es cierto.
\end{remark}

\begin{theorem}
  Sea $f: U \to \BC$ una funci\'on anal\'itica y $p \in U$ tal que $f(p) = 0$. Si
  existe una sucesi\'on $\{z_n\}_{n \geq 0}$ en $U \setminus \{p\}$ tal que
  \[
    \lim_{n \to \infty} z_n = p \text{ y } f(z_n) = 0 \text{ para todo $n$},
  \]
  entonces $f = 0$ en una vecindad abierta de $p$.
\end{theorem}

\begin{theorem}
  Sea $f: U \to \BC$ una funci\'on anal\'itica. Consideremos el conjunto
  \[
    \mathcal{V}_f = \{z \in U : f = 0 \text{ en una vecindad abierta de } z\},
  \]
  luego $\mathcal{V}_f$ es cerrado en $U$.
\end{theorem}

Decimos que un conjunto $U$ es \textbf{discreto} cuando para todo $u \in U$ existe
una vecindad $V$ de $u$ tal que $V \cap U = \{u\}$.

\begin{theorem}
  Supongamos que $U$ es conexo. Demostrar que si la funci\'on anal\'itica
  $f: U \to \BC$ no es ind\'enticamente $0$ entonces el conjunto
  \[
    \mathcal{Z}_f = \{z \in U: f(z) = 0\}  
  \]
  es cerrado en $U$ y discreto.
\end{theorem}

\begin{theorem}
  El anillo de funciones anal\'iticas en un abierto conexo $U$ es un dominio de integridad.
\end{theorem}

\begin{theorem}[Principio de continuaci\'on anal\'itica]
  Si $f, g: U \to \BC$ son dos funciones anal\'iticas tales que $f = g$ en un abierto de $U$
  entonces $f = g$ en todo $U$.
\end{theorem}