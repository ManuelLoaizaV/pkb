\subsection{Funciones Anal\'iticas}

Sea $U \subset \BC$ abierto. Decimos que $f: U \to \BC$ es \textbf{anal\'itica} si para todo $z_0 \in U$
existe un abierto $U_{z_0}$ tal que $z_0 \in U_{z_0} \subset U$ y una serie de potencias formal
$\sum_{k = 0}^{\infty} a_k z^k$ tal que
\[
  f(z) = \sum_{k = 0}^{\infty} a_k (z - z_0)^k
\]
para todo $z \in U_{z_0}$.

\begin{theorem}
  Toda funci\'on anal\'itica es holomorfa.
\end{theorem}

\begin{theorem}
  La derivada de una funci\'on anal\'itica es tambi\'en anal\'itica.
\end{theorem}

\begin{remark}
  Toda funci\'on anal\'itica es real anal\'itica pero la inversa no es cierto.
\end{remark}

\begin{theorem}
  Sea $f: U \to \BC$ una funci\'on anal\'itica y $p \in U$ tal que $f(p) = 0$. Si
  existe una sucesi\'on $\{z_n\}_{n \geq 0}$ en $U \setminus \{p\}$ tal que
  \[
    \lim_{n \to \infty} z_n = p \text{ y } f(z_n) = 0 \text{ para todo $n$},
  \]
  entonces $f = 0$ en una vecindad abierta de $p$.
\end{theorem}

\begin{theorem}
  Sea $f: U \to \BC$ una funci\'on anal\'itica. Consideremos el conjunto
  \[
    \mathcal{V}_f = \{z \in U : f = 0 \text{ en una vecindad abierta de } z\},
  \]
  luego $\mathcal{V}_f$ es cerrado en $U$.
\end{theorem}

Decimos que un conjunto $U$ es \textbf{discreto} cuando para todo $u \in U$ existe
una vecindad $V$ de $u$ tal que $V \cap U = \{u\}$.

\begin{theorem}
  Supongamos que $U$ es conexo. Demostrar que si la funci\'on anal\'itica
  $f: U \to \BC$ no es ind\'enticamente $0$ entonces el conjunto
  \[
    \mathcal{Z}_f = \{z \in U: f(z) = 0\}  
  \]
  es cerrado en $U$ y discreto.
\end{theorem}

\begin{theorem}
  El anillo de funciones anal\'iticas en un abierto conexo $U$ es un dominio de integridad.
\end{theorem}

\begin{theorem}[Principio de continuaci\'on anal\'itica]
  Si $f, g: U \to \BC$ son dos funciones anal\'iticas tales que $f = g$ en un abierto de $U$
  entonces $f = g$ en todo $U$.
\end{theorem}

Decimos que una funci\'on $f: U \to \BC$ es \textbf{abierta} si todo abierto $V \subset U$ tiene imagen
$f(V)$ abierta.

\begin{theorem}
  Una funci\'on $f: U \to \BC$ es abierta si y solo si para todo $z_0 \in U$ existe
  $\varepsilon > 0$ tal que $\{z \in \BC: |z - z_0| < \varepsilon\} \subset U$ y
  $\{f(z): |z - z_0| < \varepsilon\}$ es un conjunto de abierto de $\BC$.
\end{theorem}

\begin{example}
  Sea $k \geq 1$. La funci\'on $f: \BC \to \BC$ definida por $f(z) = z^k$ es una funci\'on abierta.
\end{example}

\begin{theorem}[Teorema de la funci\'on abierta]
  Sea $f: U \to \BC$ una funci\'on anal\'itica en $U$ conexo.
  Si $f$ no es constante entonces $f$ es una funci\'on abierta.
\end{theorem}

Algunas consecuencias importantes de este teorema son las siguientes:

\begin{theorem}[Principio del m\'odulo m\'aximo]
  Sea $f: U \to \BC$ una funci\'on anal\'itica en $U$ conexo y $z_0 \in U$.
  Si $|f(z_0)| \geq |f(z)|$ para todo $z$ en una vecindad abierta de $z_0$,
  entonces $f$ es constante.
\end{theorem}

\begin{remark}
  En el teorema anterior, basta con probar que $\Re(f(z_0)) \geq \Re(f(z))$ para todo $z$ en
  una vecindad abierta de $z_0$ para concluir que $f$ es constante.
\end{remark}

\begin{theorem}
  Sea $f: U \to \BC$ una funci\'on anal\'itica en $U$ conexo. Supongamos que $f$ es una
  isometr\'ia, es decir, $|f'(z)| = 1$ para todo $z \in U$. Existen $\alpha, \beta \in \BC$ tales que
  $|\beta| = 1$ y $f(z) = \alpha + \beta z$ para todo $z \in U$.
\end{theorem}

Denotamos por $\BD$ al disco abierto de centro en $0$ y de radio $1$, es decir,
$\BD = \{z \in \BC: |z| < 1\}$.

\begin{example}
  Sea $f: \BD \to \BD$ una funci\'on anal\'itica tal que $f(0) = 0$ y $f'(0) = 1$, entonces
  $f(z) =  z$ para todo $z \in \BD$.
\end{example}