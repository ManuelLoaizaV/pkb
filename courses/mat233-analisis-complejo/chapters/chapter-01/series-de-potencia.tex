\subsection{Series de Potencias Convergentes}

Tomemos como nuestros caballos de guerra las siguientes tres nociones:
\begin{itemize}
  \item Si $0 \leq \rho < 1$ entonces $\lim_{n \to \infty} \sum_{k = 0}^n \rho^k = 1 / (1 - \rho)$.
  \item El espacio $\BC$ es un espacio m\'etrico completo; es decir, toda sucesi\'on de Cauchy converge.
  \item $|x + y| \leq |x| + |y|$, para todo $x, y \in \BC$.
\end{itemize}

Dada una serie de potencias formal $f(z) = \sum_{k = 0}^{\infty} a_k z^k$ y un n\'umero $z_0 \in \BC$,
decimos que la serie $f$ converge en $z_0$ si
\[
  \lim_{n \to \infty} \sum_{k = 0}^n a_k z_0^k
\]
existe.

\begin{theorem}
  Si $f$ converge en $z_0$ entonces
  \[
    \lim_{n \to \infty} a_n z_0^n = 0.  
  \]
\end{theorem}

\begin{theorem}
  Si $z_0 \in \BC$ es tal que la sucesi\'on $\{a_n z_0^n\}_{n \geq 0}$ est\'a acotada, entonces
  $f$ converge en todo punto de $\{z \in \BC : |z| < |z_0|\}$.
\end{theorem}

Sea $f(z) = \sum_{k = 0}^{\infty} a_k z^k$ una serie de potencias formal, definimos el
\textbf{radio de convergencia} a $R_f \in [0, \infty]$ como
\[
  R_f = \sup\left\{r \geq 0 : \lim_{n \to \infty}\sum_{k = 0}^n |a_k| r^k\, \text{existe}\right\}.
\]

\begin{theorem}
  Sea
  \[
    C_f = \{z \in \BC : f \text{ converge en } z\}.
  \]
  Si $C_f \neq \BC$, entonces existe alg\'un $R \geq 0$ tal que
  \[
    \{z \in \BC : |z| < R\} \subset C_f \subset \{z \in \BC : |z| \leq R\}.
  \]
  Es decir, el conjunto en el que converge una serie es pr\'acticamente un disco.
\end{theorem}

\begin{theorem}[Convergencia uniforme en compactos]
  Sea $f < R_f$ y para cada $n \geq 0$ definamos la funci\'on
  $S_n: \{z \in \BC : |z| \leq z\} \to \BC$ como
  \[
    S_n(z) = \sum_{k = 0}^n a_k z^k.  
  \]
  Sea $S: \{z \in \BC : |z| \leq z\} \to \BC$ con $S(z) = \lim_{n \to \infty} S_n(z)$.
  Se cumple que
  \[
    \lim_{n \to \infty} \left(\sup_{|x| \leq r} |S_n(z) - S(z)|\right) = 0.
  \]
\end{theorem}

\begin{theorem}
  Sea $f(z) = \sum_{k = 0}^{\infty} a_k z^k$ una serie de potencias formal y consideremos la
  serie de potencias formal $f'(z) = \sum_{k = 0}^{\infty} (k + 1) a_{k + 1} z^k$. Luego,
  $R_f = R_{f'}$.
\end{theorem}

\begin{theorem}[Derivada]
  Sea $f(z) = \sum_{k = 0}^{\infty} a_k z^k$ una serie de potencias formal. Consideremos la funci\'on
  $F: \{z \in \BC : |z| < R_f\} \to \BC$ definida por
  \[
    F(z) = \lim_{n \to \infty} \sum_{k = 0}^n a_k z^k 
  \]
  y la funci\'on $G: \{z \in \BC : |z| < R_f\}$ definida por
  \[
    G(z) = \lim_{n \to \infty} \sum_{k = 0}^n (k + 1) a_{k + 1}z^k.
  \]
  Entonces $F$ es holomorfa y $F' = G$.
\end{theorem}

 \begin{example}
  La serie de potencias $\mathcal{E} = \sum_{k = 0}^{\infty} (1 / k!) z^k$ tiene un radio de convergencia infinito.
  A la suma de esta serie se le suele llamar funci\'on exponencial
  \[
    \exp(z) = \sum_{k = 0}^{\infty} \frac{z^k}{k!}.
  \]
 \end{example}

 \begin{example}
   La serie de potencias $\mathcal{L} = \sum_{k = 1}^{\infty} ((-1)^{k + 1} / k) z^k$ tiene un radio
   de convergencia $1$.
   A la suma de esta serie se le denota por
   \[
     \log(1 + z) = \sum_{k = 1}^{\infty} \frac{(-1)^{k + 1}}{k} z^k.
   \]
 \end{example}