\subsection{Series de Potencias Convergentes}

Tomemos como nuestros caballos de guerra las siguientes tres nociones:
\begin{itemize}
  \item Si $0 \leq \rho < 1$ entonces $\lim_{n \to \infty} \sum_{k = 0}^n \rho^k = 1 / (1 - \rho)$.
  \item El espacio $\BC$ es un espacio m\'etrico completo; es decir, toda sucesi\'on de Cauchy converge.
  \item $|x + y| \leq |x| + |y|$, para todo $x, y \in \BC$.
\end{itemize}

Dada una serie de potencias formal $f(z) = \sum_{k = 0}^{\infty} a_k z^k$ y un n\'umero $z_0 \in \BC$,
decimos que la serie $f$ converge en $z_0$ si
\[
  \lim_{n \to \infty} \sum_{k = 0}^n a_k z_0^k
\]
existe.

\begin{theorem}
  Si $f$ converge en $z_0$ entonces
  \[
    \lim_{n \to \infty} a_n z_0^n = 0.  
  \]
\end{theorem}

\begin{theorem}
  Si $z_0 \in \BC$ es tal que la sucesi\'on $\{a_n z_0^n\}_{n \geq 0}$ est\'a acotada, entonces
  $f$ converge en todo punto de $\{z \in \BC : |z| < |z_0|\}$.
\end{theorem}

\begin{theorem}

\end{theorem}