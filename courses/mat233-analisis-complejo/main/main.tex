\documentclass{article}
\usepackage[utf8]{inputenc}
\usepackage{amsfonts,latexsym,amsthm,amssymb,amsmath,amscd,euscript}
\usepackage{mathtools}
\usepackage{esint}
\usepackage{framed}
% Descomentar fullpage cuando se quiera utilizar menos margen horizontal
%\usepackage{fullpage}
\usepackage{hyperref}
    \hypersetup{colorlinks=true,citecolor=blue,urlcolor =black,linkbordercolor={1 0 0}}

\newenvironment{statement}[1]{\smallskip\noindent\color[rgb]{1.00,0.00,0.50} {\bf #1.}}{}
\allowdisplaybreaks[1]

% Comandos para teoremas, definiciones, ejemplos, lemas, etc. para sus respectivos body types.
\renewcommand*{\proofname}{Prueba}
\renewcommand{\contentsname}{Contenido}

\newtheorem{theorem}{Teorema}
\newtheorem*{proposition}{Proposici\'on}
\newtheorem{lemma}[theorem]{Lema}
\newtheorem{corollary}[theorem]{Corolario}
\newtheorem{conjecture}[theorem]{Conjetura}
\newtheorem*{postulate}{Postulado}
\theoremstyle{definition}
\newtheorem{defn}[theorem]{Definici\'on}
\newtheorem{example}[theorem]{Ejemplo}

\theoremstyle{remark}
\newtheorem*{remark}{Observaci\'on}
\newtheorem*{notation}{Notaci\'on}
\newtheorem*{note}{Nota}

% Define tus comandos para hacer la vida más fácil.
\newcommand{\BR}{\mathbb R}
\newcommand{\BC}{\mathbb C}
\newcommand{\BF}{\mathbb F}
\newcommand{\BQ}{\mathbb Q}
\newcommand{\BZ}{\mathbb Z}
\newcommand{\BN}{\mathbb N}

\title{MAT230 Ecuaciones Diferenciales Parciales}
\author{Manuel Loaiza Vasquez}
\date{Septiembre 2021}

\begin{document}

\maketitle

\vspace*{-0.25in}
\centerline{Pontificia Universidad Cat\'olica del Per\'u}
\centerline{Lima, Per\'u}
\centerline{\href{mailto:manuel.loaiza@pucp.edu.pe}{{\tt manuel.loaiza@pucp.edu.pe}}}
\vspace*{0.15in}

\begin{framed}
    Resumen te\'orico del curso de An\'alisis Complejo dictado por el profesor
    David Garc\'ia.
\end{framed}

\tableofcontents

\newpage

\section{Funciones Holomorfas}

\subsection{Diferenciaci\'on Compleja}

Sea $f: U \to \BC$ una funci\'on compleja. Para un $z_0 \in U$, definimos la \textbf{derivada}
en $z_0$ a
\[
  \lim_{h \to 0} \frac{f(z_0 + h) - f(z_0)}{h}.
\] 
Notemos que este l\'imite puede no existir. Cuando existe decimos que $f$ es \textbf{diferenciable}
en $z_0$.

¿A qu\'e nos referimos con un l\'imite ``complejo" $h \to 0$?
Es lo que nuestra intuici\'on nos dir\'ia:
para todo $\varepsilon > 0$ debe existir un $\delta > 0$ tal que
\[
  0 < |h| < \delta \implies \left|\frac{f(z_0 + h) - f(z_0)}{h} - L\right| < \varepsilon.
\]

\begin{example}
  Sea $f(z) = \overline{z}$ una conjugaci\'on compleja, $f: \BC \to \BC$. Esta funci\'on, a pesar
  de su naturaleza, no es holomorfa. De hecho, en $z = 0$ tenemos
  \[
    \frac{f(h) - f(0)}{h} = \frac{\overline{h}}{h}.
  \]
  Esto no tiene un l\'imite cuando $h \to 0$, pues dependiendo de ``la direcci\'on" en la que
  nos acercamos a cero podemos tener diferentes valores.
\end{example}

Si una funci\'on $f: U \to \BC$ es diferenciable compleja en todos sus puntos de su dominio
es llamada \textbf{holomorfa}. En el caso particular de una funci\'on holomorfa con dominio
$U = \BC$, llamamos a esta funci\'on \textbf{entera}.

\begin{example}
  En todos los ejemplos de abajo, la derivada de la funci\'on es la misma que su an\'alogo real.
  \begin{itemize}
    \item Todo polinomio $z \mapsto z^n + c_{n - 1} z^{n - 1} + \cdots + c_0$ es holomorfo.
    \item La exponencial compleja $\exp: x + yi \mapsto e^x (\cos y + i \sin y)$ es holomorfa.
    \item $\sin$ y $\cos$ son holomorfos cuando se extienden al plano complejo como
    $\cos z = (e^{iz} + e^{-iz}) / 2$ y $\sin z = (e^{iz} - e^{-iz}) / 2i$.
    \item Como siempre, la suma, producto, regla de la cadena y otras aplican. De esta manera, la suma,
    producto, divisi\'on distinta de cero, y composici\'on de funciones holomorfas son tambi\'en
    holomorfas.
  \end{itemize}
\end{example}

\section{Teor\'ia de Cauchy}

\section{Otras Versiones del Teorema de Cauchy}

\section{T\'opicos Especiales}

\end{document}