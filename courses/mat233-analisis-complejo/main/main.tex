\documentclass{article}
\usepackage[utf8]{inputenc}
\usepackage{amsfonts,latexsym,amsthm,amssymb,amsmath,amscd,euscript}
\usepackage{mathtools}
\usepackage{esint}
\usepackage{framed}
% Descomentar fullpage cuando se quiera utilizar menos margen horizontal
%\usepackage{fullpage}
\usepackage{hyperref}
    \hypersetup{colorlinks=true,citecolor=blue,urlcolor =black,linkbordercolor={1 0 0}}

\newenvironment{statement}[1]{\smallskip\noindent\color[rgb]{1.00,0.00,0.50} {\bf #1.}}{}
\allowdisplaybreaks[1]

% Comandos para teoremas, definiciones, ejemplos, lemas, etc. para sus respectivos body types.
\renewcommand*{\proofname}{Prueba}
\renewcommand{\contentsname}{Contenido}

\newtheorem{theorem}{Teorema}
\newtheorem*{proposition}{Proposici\'on}
\newtheorem{lemma}[theorem]{Lema}
\newtheorem{corollary}[theorem]{Corolario}
\newtheorem{conjecture}[theorem]{Conjetura}
\newtheorem*{postulate}{Postulado}
\theoremstyle{definition}
\newtheorem{defn}[theorem]{Definici\'on}
\newtheorem{example}[theorem]{Ejemplo}

\theoremstyle{remark}
\newtheorem*{remark}{Observaci\'on}
\newtheorem*{notation}{Notaci\'on}
\newtheorem*{note}{Nota}

% Define tus comandos para hacer la vida más fácil.
\newcommand{\BR}{\mathbb R}
\newcommand{\BC}{\mathbb C}
\newcommand{\BD}{\mathbb{D}}
\newcommand{\BF}{\mathbb F}
\newcommand{\BQ}{\mathbb Q}
\newcommand{\BZ}{\mathbb Z}
\newcommand{\BN}{\mathbb N}

\title{MAT233 An\'alisis Complejo}
\author{Manuel Loaiza Vasquez}
\date{Septiembre 2021}

\begin{document}

\maketitle

\vspace*{-0.25in}
\centerline{Pontificia Universidad Cat\'olica del Per\'u}
\centerline{Lima, Per\'u}
\centerline{\href{mailto:manuel.loaiza@pucp.edu.pe}{{\tt manuel.loaiza@pucp.edu.pe}}}
\vspace*{0.15in}

\begin{framed}
    Resumen te\'orico del curso de An\'alisis Complejo dictado por el profesor
    David Garc\'ia.
\end{framed}

\tableofcontents

\newpage

\section{Funciones Holomorfas}

\subsection{Diferenciaci\'on Compleja}

Sea $f: U \to \BC$ una funci\'on compleja. Para un $z_0 \in U$, definimos la \textbf{derivada}
en $z_0$ a
\[
  \lim_{h \to 0} \frac{f(z_0 + h) - f(z_0)}{h}.
\] 
Notemos que este l\'imite puede no existir. Cuando existe decimos que $f$ es \textbf{diferenciable}
en $z_0$.

¿A qu\'e nos referimos con un l\'imite ``complejo" $h \to 0$?
Es lo que nuestra intuici\'on nos dir\'ia:
para todo $\varepsilon > 0$ debe existir un $\delta > 0$ tal que
\[
  0 < |h| < \delta \implies \left|\frac{f(z_0 + h) - f(z_0)}{h} - L\right| < \varepsilon.
\]

\begin{example}
  Sea $f(z) = \overline{z}$ una conjugaci\'on compleja, $f: \BC \to \BC$. Esta funci\'on, a pesar
  de su naturaleza, no es holomorfa. De hecho, en $z = 0$ tenemos
  \[
    \frac{f(h) - f(0)}{h} = \frac{\overline{h}}{h}.
  \]
  Esto no tiene un l\'imite cuando $h \to 0$, pues dependiendo de ``la direcci\'on" en la que
  nos acercamos a cero podemos tener diferentes valores.
\end{example}

Si una funci\'on $f: U \to \BC$ es diferenciable compleja en todos sus puntos de su dominio
es llamada \textbf{holomorfa}. En el caso particular de una funci\'on holomorfa con dominio
$U = \BC$, llamamos a esta funci\'on \textbf{entera}.

\begin{example}
  En todos los ejemplos de abajo, la derivada de la funci\'on es la misma que su an\'alogo real.
  \begin{itemize}
    \item Todo polinomio $z \mapsto z^n + c_{n - 1} z^{n - 1} + \cdots + c_0$ es holomorfo.
    \item La exponencial compleja $\exp: x + yi \mapsto e^x (\cos y + i \sin y)$ es holomorfa.
    \item $\sin$ y $\cos$ son holomorfos cuando se extienden al plano complejo como
    $\cos z = (e^{iz} + e^{-iz}) / 2$ y $\sin z = (e^{iz} - e^{-iz}) / 2i$.
    \item Como siempre, la suma, producto, regla de la cadena y otras aplican. De esta manera, la suma,
    producto, divisi\'on distinta de cero, y composici\'on de funciones holomorfas son tambi\'en
    holomorfas.
  \end{itemize}
\end{example}

\subsection{Series de Potencias Convergentes}

Tomemos como nuestros caballos de guerra las siguientes tres nociones:
\begin{itemize}
  \item Si $0 \leq \rho < 1$ entonces $\lim_{n \to \infty} \sum_{k = 0}^n \rho^k = 1 / (1 - \rho)$.
  \item El espacio $\BC$ es un espacio m\'etrico completo; es decir, toda sucesi\'on de Cauchy converge.
  \item $|x + y| \leq |x| + |y|$, para todo $x, y \in \BC$.
\end{itemize}

Dada una serie de potencias formal $f(z) = \sum_{k = 0}^{\infty} a_k z^k$ y un n\'umero $z_0 \in \BC$,
decimos que la serie $f$ converge en $z_0$ si
\[
  \lim_{n \to \infty} \sum_{k = 0}^n a_k z_0^k
\]
existe.

\begin{theorem}
  Si $f$ converge en $z_0$ entonces
  \[
    \lim_{n \to \infty} a_n z_0^n = 0.  
  \]
\end{theorem}

\begin{theorem}
  Si $z_0 \in \BC$ es tal que la sucesi\'on $\{a_n z_0^n\}_{n \geq 0}$ est\'a acotada, entonces
  $f$ converge en todo punto de $\{z \in \BC : |z| < |z_0|\}$.
\end{theorem}

Sea $f(z) = \sum_{k = 0}^{\infty} a_k z^k$ una serie de potencias formal, definimos el
\textbf{radio de convergencia} a $R_f \in [0, \infty]$ como
\[
  R_f = \sup\left\{r \geq 0 : \lim_{n \to \infty}\sum_{k = 0}^n |a_k| r^k\, \text{existe}\right\}.
\]

\begin{theorem}
  Sea
  \[
    C_f = \{z \in \BC : f \text{ converge en } z\}.
  \]
  Si $C_f \neq \BC$, entonces existe alg\'un $R \geq 0$ tal que
  \[
    \{z \in \BC : |z| < R\} \subset C_f \subset \{z \in \BC : |z| \leq R\}.
  \]
  Es decir, el conjunto en el que converge una serie es pr\'acticamente un disco.
\end{theorem}

\begin{theorem}[Convergencia uniforme en compactos]
  Sea $f < R_f$ y para cada $n \geq 0$ definamos la funci\'on
  $S_n: \{z \in \BC : |z| \leq z\} \to \BC$ como
  \[
    S_n(z) = \sum_{k = 0}^n a_k z^k.  
  \]
  Sea $S: \{z \in \BC : |z| \leq z\} \to \BC$ con $S(z) = \lim_{n \to \infty} S_n(z)$.
  Se cumple que
  \[
    \lim_{n \to \infty} \left(\sup_{|x| \leq r} |S_n(z) - S(z)|\right) = 0.
  \]
\end{theorem}

\begin{theorem}
  Sea $f(z) = \sum_{k = 0}^{\infty} a_k z^k$ una serie de potencias formal y consideremos la
  serie de potencias formal $f'(z) = \sum_{k = 0}^{\infty} (k + 1) a_{k + 1} z^k$. Luego,
  $R_f = R_{f'}$.
\end{theorem}

\begin{theorem}[Derivada]
  Sea $f(z) = \sum_{k = 0}^{\infty} a_k z^k$ una serie de potencias formal. Consideremos la funci\'on
  $F: \{z \in \BC : |z| < R_f\} \to \BC$ definida por
  \[
    F(z) = \lim_{n \to \infty} \sum_{k = 0}^n a_k z^k 
  \]
  y la funci\'on $G: \{z \in \BC : |z| < R_f\}$ definida por
  \[
    G(z) = \lim_{n \to \infty} \sum_{k = 0}^n (k + 1) a_{k + 1}z^k.
  \]
  Entonces $F$ es holomorfa y $F' = G$.
\end{theorem}

 \begin{example}
  La serie de potencias $\mathcal{E} = \sum_{k = 0}^{\infty} (1 / k!) z^k$ tiene un radio de convergencia infinito.
  A la suma de esta serie se le suele llamar funci\'on exponencial
  \[
    \exp(z) = \sum_{k = 0}^{\infty} \frac{z^k}{k!}.
  \]
 \end{example}

 \begin{example}
   La serie de potencias $\mathcal{L} = \sum_{k = 1}^{\infty} ((-1)^{k + 1} / k) z^k$ tiene un radio
   de convergencia $1$.
   A la suma de esta serie se le denota por
   \[
     \log(1 + z) = \sum_{k = 1}^{\infty} \frac{(-1)^{k + 1}}{k} z^k.
   \]
 \end{example}

\subsection{Funciones Anal\'iticas}

Sea $U \subset \BC$ abierto. Decimos que $f: U \to \BC$ es \textbf{anal\'itica} si para todo $z_0 \in U$
existe un abierto $U_{z_0}$ tal que $z_0 \in U_{z_0} \subset U$ y una serie de potencias formal
$\sum_{k = 0}^{\infty} a_k z^k$ tal que
\[
  f(z) = \sum_{k = 0}^{\infty} a_k (z - z_0)^k
\]
para todo $z \in U_{z_0}$.

\begin{theorem}
  Toda funci\'on anal\'itica es holomorfa.
\end{theorem}

\begin{theorem}
  La derivada de una funci\'on anal\'itica es tambi\'en anal\'itica.
\end{theorem}

\begin{remark}
  Toda funci\'on anal\'itica es real anal\'itica pero la inversa no es cierto.
\end{remark}

\begin{theorem}
  Sea $f: U \to \BC$ una funci\'on anal\'itica y $p \in U$ tal que $f(p) = 0$. Si
  existe una sucesi\'on $\{z_n\}_{n \geq 0}$ en $U \setminus \{p\}$ tal que
  \[
    \lim_{n \to \infty} z_n = p \text{ y } f(z_n) = 0 \text{ para todo $n$},
  \]
  entonces $f = 0$ en una vecindad abierta de $p$.
\end{theorem}

\begin{theorem}
  Sea $f: U \to \BC$ una funci\'on anal\'itica. Consideremos el conjunto
  \[
    \mathcal{V}_f = \{z \in U : f = 0 \text{ en una vecindad abierta de } z\},
  \]
  luego $\mathcal{V}_f$ es cerrado en $U$.
\end{theorem}

Decimos que un conjunto $U$ es \textbf{discreto} cuando para todo $u \in U$ existe
una vecindad $V$ de $u$ tal que $V \cap U = \{u\}$.

\begin{theorem}
  Supongamos que $U$ es conexo. Demostrar que si la funci\'on anal\'itica
  $f: U \to \BC$ no es ind\'enticamente $0$ entonces el conjunto
  \[
    \mathcal{Z}_f = \{z \in U: f(z) = 0\}  
  \]
  es cerrado en $U$ y discreto.
\end{theorem}

\begin{theorem}
  El anillo de funciones anal\'iticas en un abierto conexo $U$ es un dominio de integridad.
\end{theorem}

\begin{theorem}[Principio de continuaci\'on anal\'itica]
  Si $f, g: U \to \BC$ son dos funciones anal\'iticas tales que $f = g$ en un abierto de $U$
  entonces $f = g$ en todo $U$.
\end{theorem}

Decimos que una funci\'on $f: U \to \BC$ es \textbf{abierta} si todo abierto $V \subset U$ tiene imagen
$f(V)$ abierta.

\begin{theorem}
  Una funci\'on $f: U \to \BC$ es abierta si y solo si para todo $z_0 \in U$ existe
  $\varepsilon > 0$ tal que $\{z \in \BC: |z - z_0| < \varepsilon\} \subset U$ y
  $\{f(z): |z - z_0| < \varepsilon\}$ es un conjunto de abierto de $\BC$.
\end{theorem}

\begin{example}
  Sea $k \geq 1$. La funci\'on $f: \BC \to \BC$ definida por $f(z) = z^k$ es una funci\'on abierta.
\end{example}

\begin{theorem}[Teorema de la funci\'on abierta]
  Sea $f: U \to \BC$ una funci\'on anal\'itica en $U$ conexo.
  Si $f$ no es constante entonces $f$ es una funci\'on abierta.
\end{theorem}

Algunas consecuencias importantes de este teorema son las siguientes:

\begin{theorem}[Principio del m\'odulo m\'aximo]
  Sea $f: U \to \BC$ una funci\'on anal\'itica en $U$ conexo y $z_0 \in U$.
  Si $|f(z_0)| \geq |f(z)|$ para todo $z$ en una vecindad abierta de $z_0$,
  entonces $f$ es constante.
\end{theorem}

\begin{remark}
  En el teorema anterior, basta con probar que $\Re(f(z_0)) \geq \Re(f(z))$ para todo $z$ en
  una vecindad abierta de $z_0$ para concluir que $f$ es constante.
\end{remark}

\begin{theorem}
  Sea $f: U \to \BC$ una funci\'on anal\'itica en $U$ conexo. Supongamos que $f$ es una
  isometr\'ia, es decir, $|f'(z)| = 1$ para todo $z \in U$. Existen $\alpha, \beta \in \BC$ tales que
  $|\beta| = 1$ y $f(z) = \alpha + \beta z$ para todo $z \in U$.
\end{theorem}

Denotamos por $\BD$ al disco abierto de centro en $0$ y de radio $1$, es decir,
$\BD = \{z \in \BC: |z| < 1\}$.

\begin{example}
  Sea $f: \BD \to \BD$ una funci\'on anal\'itica tal que $f(0) = 0$ y $f'(0) = 1$, entonces
  $f(z) =  z$ para todo $z \in \BD$.
\end{example}

\section{Teor\'ia de Cauchy}

\section{Otras Versiones del Teorema de Cauchy}

\section{T\'opicos Especiales}

\end{document}