\documentclass{article}
\usepackage[utf8]{inputenc}
\usepackage{amsfonts,latexsym,amsthm,amssymb,amsmath,amscd,euscript}
\usepackage{mathtools}
\usepackage{framed}
% Descomentar fullpage cuando se quiera utilizar menos margen horizontal
%\usepackage{fullpage}
\usepackage{hyperref}
    \hypersetup{colorlinks=true,citecolor=blue,urlcolor =black,linkbordercolor={1 0 0}}

\newenvironment{statement}[1]{\smallskip\noindent\color[rgb]{1.00,0.00,0.50} {\bf #1.}}{}
\allowdisplaybreaks[1]

% Comandos para teoremas, definiciones, ejemplos, lemas, etc. para sus respectivos body types.
\renewcommand*{\proofname}{Prueba}
\renewcommand{\contentsname}{Contenido}

\newtheorem{theorem}{Teorema}
\newtheorem*{proposition}{Proposici\'on}
\newtheorem{lemma}[theorem]{Lema}
\newtheorem{corollary}[theorem]{Corolario}
\newtheorem{conjecture}[theorem]{Conjetura}
\newtheorem*{postulate}{Postulado}
\theoremstyle{definition}
\newtheorem{defn}[theorem]{Definici\'on}
\newtheorem{example}[theorem]{Ejemplo}

\theoremstyle{remark}
\newtheorem*{remark}{Observaci\'on}
\newtheorem*{notation}{Notaci\'on}
\newtheorem*{note}{Nota}

% Define tus comandos para hacer la vida más fácil.
\newcommand{\BR}{\mathbb R}
\newcommand{\BC}{\mathbb C}
\newcommand{\BF}{\mathbb F}
\newcommand{\BQ}{\mathbb Q}
\newcommand{\BZ}{\mathbb Z}
\newcommand{\BN}{\mathbb N}

\title{MAT230 Ecuaciones Diferenciales Parciales}
\author{Manuel Loaiza Vasquez}
\date{Septiembre 2021}

\begin{document}

\maketitle

\vspace*{-0.25in}
\centerline{Pontificia Universidad Cat\'olica del Per\'u}
\centerline{Lima, Per\'u}
\centerline{\href{mailto:manuel.loaiza@pucp.edu.pe}{{\tt manuel.loaiza@pucp.edu.pe}}}
\vspace*{0.15in}

\begin{framed}
    Resumen te\'orico del curso de Ecuaciones Diferenciales Parciales dictado por el profesor
    Jonathan Farf\'an Vargas.
\end{framed}

\tableofcontents

\newpage

\section{Cuatro EDPs Lineales Importantes}

\subsection{Ecuaci\'on de Transporte}

\[
  u_t + b \cdot D u = 0,\, (x, t) \in \BR^n \times [0, \infty).  
\]

\subsection{Ecuaci\'on de Laplace}

Sea $U$ un abierto en $\BR^n$, $f: U \to \BR$.
Queremos encontrar una funci\'on $u: \bar U \to \BR$ tal que

\begin{itemize}
  \item $\Delta u = 0$ (Laplace)
  \item $\Delta u = f$ (Poisson)
\end{itemize}

\begin{defn}[Funci\'on Arm\'onica]
  Una funci\'on $u$ de clase $C^2$ tal que $\Delta u = 0$ es llamada
  funci\'on arm\'onica.
\end{defn}

\begin{defn}
  La funci\'on
  \[
    \Phi (x) = \begin{cases}
      -\frac{1}{2 \pi} \log|x| & n = 2\\
      \frac{1}{n(n - 2)\alpha(n)|x|^{n - 2}} & n \geq 3,
    \end{cases} 
  \]
  definida para $x \in \BR^n$, $x \neq 0$, es la soluci\'on fundamental de la ecuaci\'on de Laplace.
\end{defn}

\begin{theorem}
  Definamos
  \[
    u(x) = \int_{\BR^n} \Phi(x - y) f(y)\, dy = \begin{cases}
      -\frac{1}{2} \int_{\BR^2} \log(|x - y|)f(y)\, dy & n = 2\\
      \frac{1}{n(n - 2)\alpha(n)} \int_{\BR^n} \frac{f(y)}{|x - y|^{n - 2}}\, dy & n \geq 3.
    \end{cases}
  \]
  Luego
  \begin{enumerate}
    \item $u \in C^2(\BR^n)$
    \item $-\Delta u = f$ en $\BR^n$.
  \end{enumerate}
\end{theorem}

\begin{theorem}[Mean-value formulas for Laplace's equation]
  Si $u \in C^2(U)$ es arm\'onica, luego
  \[
    u(x) = \fint_{\partial B(x, r)} u\, dS = \fint_{B(x, r)} u\, dy
  \]
  para cada bola $B(x, r) \subset U$.
\end{theorem}

\begin{theorem}[Converse to mean-value property]
  Si $u \in C^2(U)$ satisface
  \[
    u(x) = \fint_{\partial B(x, r)} u\, dS
  \]
  para cada bola $B(x, r) \subset U$, luego $u$ es arm\'onica.
\end{theorem}

Sea $U \subset \BR^n$ abierto y acotado, obtenemos las siguientes propiedades
de las funciones arm\'onicas basadas en las f\'ormulas anteriores:
\begin{enumerate}
  \item Strong maximum principle, uniqueness.
  \item Regularity.
  \item Local estimates for harmonic functions.
  \item Liouville's Theorem.
  \item Analyticity.
\end{enumerate}

\begin{theorem}[Strong maximum principle]
  Supongamos que $u \in C^2(U) \cap C(\bar{U})$ es arm\'onica en $U$. Luego
  \[
    \max_{\bar{U}}   u = \max_{\partial U} u.
  \]
  Adem\'as, si $U$ es conexo y existe un punto $x_0 \in U$ tal que
  \[
    u(x_0) = \max_{\bar{U}} u,
  \]
  entonces $u$ es constante en $U$.
\end{theorem}

\begin{theorem}[Uniqueness]
  Sea $g \in C(\partial U)$, $f \in C(U)$. Luego existe a lo m\'as una soluci\'on
  $u \in C^2 \cap C(\bar{U})$ del problema de valor en la frontera
  
\end{theorem}

\input{../chapters/chapter-01/heat-equation.tex}

\input{../chapters/chapter-01/wave-equation.tex}

\end{document}