\subsection{Ecuaci\'on de Laplace}

Sea $U$ un abierto en $\BR^n$, $f: U \to \BR$.
Queremos encontrar una funci\'on $u: \bar U \to \BR$ tal que

\begin{itemize}
  \item $\Delta u = 0$ (Laplace)
  \item $\Delta u = f$ (Poisson)
\end{itemize}

\begin{defn}[Funci\'on Arm\'onica]
  Una funci\'on $u$ de clase $C^2$ tal que $\Delta u = 0$ es llamada
  funci\'on arm\'onica.
\end{defn}

\begin{defn}
  La funci\'on
  \[
    \Phi (x) = \begin{cases}
      -\frac{1}{2 \pi} \log|x| & n = 2\\
      \frac{1}{n(n - 2)\alpha(n)|x|^{n - 2}} & n \geq 3,
    \end{cases} 
  \]
  definida para $x \in \BR^n$, $x \neq 0$, es la soluci\'on fundamental de la ecuaci\'on de Laplace.
\end{defn}

\begin{theorem}
  Definamos
  \[
    u(x) = \int_{\BR^n} \Phi(x - y) f(y)\, dy = \begin{cases}
      -\frac{1}{2} \int_{\BR^2} \log(|x - y|)f(y)\, dy & n = 2\\
      \frac{1}{n(n - 2)\alpha(n)} \int_{\BR^n} \frac{f(y)}{|x - y|^{n - 2}}\, dy & n \geq 3.
    \end{cases}
  \]
  Luego
  \begin{enumerate}
    \item $u \in C^2(\BR^n)$
    \item $-\Delta u = f$ en $\BR^n$.
  \end{enumerate}
\end{theorem}

\begin{theorem}[Mean-value formulas for Laplace's equation]
  Si $u \in C^2(U)$ es arm\'onica, luego
  \[
    u(x) = \fint_{\partial B(x, r)} u\, dS = \fint_{B(x, r)} u\, dy
  \]
  para cada bola $B(x, r) \subset U$.
\end{theorem}

\begin{theorem}[Converse to mean-value property]
  Si $u \in C^2(U)$ satisface
  \[
    u(x) = \fint_{\partial B(x, r)} u\, dS
  \]
  para cada bola $B(x, r) \subset U$, luego $u$ es arm\'onica.
\end{theorem}

Sea $U \subset \BR^n$ abierto y acotado, obtenemos las siguientes propiedades
de las funciones arm\'onicas basadas en las f\'ormulas anteriores:
\begin{enumerate}
  \item Strong maximum principle, uniqueness.
  \item Regularity.
  \item Local estimates for harmonic functions.
  \item Liouville's Theorem.
  \item Analyticity.
\end{enumerate}

\begin{theorem}[Strong maximum principle]
  Supongamos que $u \in C^2(U) \cap C(\bar{U})$ es arm\'onica en $U$. Luego
  \[
    \max_{\bar{U}}   u = \max_{\partial U} u.
  \]
  Adem\'as, si $U$ es conexo y existe un punto $x_0 \in U$ tal que
  \[
    u(x_0) = \max_{\bar{U}} u,
  \]
  entonces $u$ es constante en $U$.
\end{theorem}

\begin{theorem}[Uniqueness]
  Sea $g \in C(\partial U)$, $f \in C(U)$. Luego existe a lo m\'as una soluci\'on
  $u \in C^2 \cap C(\bar{U})$ del problema de valor en la frontera
  
\end{theorem}