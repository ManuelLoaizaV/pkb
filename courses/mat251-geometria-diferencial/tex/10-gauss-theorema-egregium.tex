\section{Gauss' Theorema Egregium}

\subsection{The Gauss and Codazzi-Mainardi equations}

\begin{proposition}[Codazzi-Mainardi Equations]
  Given the first and second fundamental forms of a surface patch
  $\sigma(u, v)$, then
  \begin{align*}
    L_v - M_u &= L \Gamma_{12}^1 + M(\Gamma_{12}^2 - \Gamma_{11}^1) - N \Gamma_{11}^2,\\
    M_v - N_u &= L \Gamma_{22}^1 + M(\Gamma_{22}^2 - \Gamma_{12}^1) - N \Gamma_{12}^2.
  \end{align*}
\end{proposition}

\begin{proposition}[Gauss Equations]
  If $K$ is the Gaussian curvature of the surface patch $\gamma(u, v)$
  in the preceding proposition, then
  \begin{align*}
    EK &= (\Gamma_{11}^2)_v - (\Gamma_{12}^2)_u + \Gamma_{11}^1 \Gamma_{12}^2 +
    \Gamma_{11}^2 \Gamma_{22}^2 - \Gamma_{12}^1 \Gamma_{11}^2 - (\Gamma_{12}^2)^2, \\
    FK &= (\Gamma_{12}^1)_u - (\Gamma_{11}^1)_v + \Gamma_{12}^2 \Gamma_{12}^1 - \Gamma_{11}^2 \Gamma_{22}^1, \\
    FK &= (\Gamma_{12}^2)_v - (\Gamma_{22}^2)_u + \Gamma_{12}^1 \Gamma_{12}^2 - \Gamma_{22}^1 \Gamma_{11}^2, \\
    GK &= (\Gamma_{22}^1)_u - (\Gamma_{12}^1)_v + \Gamma_{22}^1 \Gamma_{11}^1 +
    \Gamma_{22}^2 \Gamma_{12}^2 - (\Gamma_{12}^1)^2 - \Gamma_{12}^2 \Gamma_{22}^1. \\
  \end{align*}
\end{proposition}

\begin{theorem}
  Let $\sigma: U \to \BR^3$ and $\tilde \sigma: U \to \BR^3$ be surface patches with the same
  first and second fundamental form. Then, there is a direct isometry $M$ of $\BR^3$ such that $\tilde \sigma = M(\sigma)$.
\end{theorem}

\subsection{Gauss' remarkable theorem}

\begin{theorem}[Gauss' Theorema Egregium]
  The Gaussian curvature of a surface is preserved by local isometries.
\end{theorem}

\begin{corollary}
  The Gaussian curvature is given by
  \[
    K = \frac{
      \begin{vmatrix}
        -\frac{1}{2} E_{vv} + F_{uv} - \frac{1}{2} G_{uu} & \frac{1}{2} E_u & F_u - \frac{1}{2} E_v \\
        F_v - \frac{1}{2} G_u & E & F \\
        \frac{1}{2} G_v & F & G
      \end{vmatrix}
      -
      \begin{vmatrix}
        0 & \frac{1}{2} E_v & \frac{1}{2} G_u \\
        \frac{1}{2} E_v & E & F \\
        \frac{1}{2} G_u & F & G
      \end{vmatrix}
    }{
      (EG - F^2)^2
    }
    .
  \]
\end{corollary}

\begin{corollary}
  With the above notation
  \begin{itemize}
    \item If $F = 0$, we have
    \[
      K = -\frac{1}{2\sqrt{EG}}\left(
        \frac{\partial}{\partial u} \left(\frac{G_u}{\sqrt{EG}}\right)
        + \frac{\partial}{\partial v} \left(\frac{E_v}{\sqrt{EG}}\right)
      \right).
    \]
    \item If $E = 1$ and $F = 0$, we have
    \[
      K = - \frac{1}{\sqrt{G}} \frac{\partial^2 \sqrt{G}}{\partial u^2}.
    \]
  \end{itemize}
\end{corollary}

\begin{proposition}
  Any map of any region of the earth's surface must distort distances.
\end{proposition}

\subsection{Surfaces of constant Gaussian curvature}

\begin{theorem}
  Any point of a surface of constant Gaussian curvature is contained in a patch
  that is isometric to an open subset of a plane, a sphere or a pseudosphere.
\end{theorem}

\begin{proposition}
  Let $S$ be a surface of constant Gaussian curvature $-1$.
  If $p \in S$, there is a surface patch of $S$ containing $p$ whose
  first and second fundamental forms are
  \[
    du^2 + 2 \cos \theta du dv + dv^2 \text{ and } 2 \sin \theta du dv,  
  \]
  respectively, where $\theta(u, v)$ is a smooth function such that
  $0 < \theta < \pi$ for all $u, v$.
\end{proposition}

\begin{lemma}
  Let $\sigma: U \to \BR^3$ be a surface patch containing a point $p$ that is not an umbilic.
  Let $\kappa_1 \geq \kappa_2$ be the principal curvatures of $\sigma$ and suppose that $\kappa_1$
  has a local maximum at $p$ and $\kappa_2$ has a local minimum there.
  Then, the Gaussian curvature of $\sigma$ at $p$ is not positive.
\end{lemma}

\begin{theorem}
  Every connected compact surface whose Gaussian curvature is constant is a sphere.
\end{theorem}

\subsection{Geodesic mappings}

\begin{defn}
  If $S$ and $\tilde S$ are surfaces, a local diffeomorphism $F: S \to \tilde S$
  is said to be geodesic if $f$ takes every pre-geodesic on $S$ to a pre-geodesic on $\tilde S$.
\end{defn}

\begin{proposition}
  The following are geodesic local diffeomorphisms:
  \begin{itemize}
    \item Every local isometry.
    \item Every dilation of $\BR^3$.
    \item Every composite of geodesic local diffeomorphisms.
  \end{itemize}
\end{proposition}

\begin{theorem}
  Let $S$ be a connected surface.
  If there is a geodesic local diffeomorphism from $S$ to a plane,
  then $S$ has constant Gaussian curvature.
  Conversely, if $S$ has constant Gaussian curvature, then for any point
  $p \in S$ there is a surface patch $\sigma: U \to S$ of $S$ such that
  $p \in \sigma(U)$ and $\sigma^{-1}: \sigma(U) \to U$ is a geodesic diffeomorphism.
\end{theorem}