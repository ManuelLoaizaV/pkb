\section{The first fundamental form}

\subsection{Lengths of curves on surfaces}

\begin{defn}
  Let $p$ be a point of a surface $S$.
  The first fundamental form of $S$ at $p$
  associates to tangent vectors $v, w \in T_p S$ the scalar
  \[
    \langle v, w \rangle_{p, S} = v \cdot w.
  \]
\end{defn}

\begin{remark}
  The first fundamental form is an example of inner product on $\BR^3$.
\end{remark}

Suppose that $sigma(u, v)$ is a surface patch of $S$.
Then, any tangent vector to $S$ at a point $p$ in the image of $\sigma$
can be expressed uniquely as a linear combination of $\sigma_u$ and $\sigma_v$.
Define maps $du: T_p S \to \BR$ and $dv: T_p S \to \BR$ by
\[
  du(v) = \lambda, \, du(v) = \mu, \, v = \lambda \sigma_u + \mu \sigma_v.
\]
We have
\[
  \langle v, v \rangle = \lambda^2 \langle \sigma_u, \sigma_v \rangle +
  2 \lambda \mu \langle \sigma_u, \sigma_v \rangle +
  \mu^2 \langle \sigma_v, \sigma_v \rangle.  
\]
Writing
\[
  E = \|\sigma_u\|^2,
  F = \sigma_u \cdot \sigma_v,
  G = \|\sigma_v\|^2,
\]
this becomes
\[
  \langle v, v \rangle =
  E du^2 + 2 F du\, dv   + G dv^2
\]
which is called the first fundamental form of the surface patch $\sigma(u, v)$.

If $\gamma$ is a curve lying in the image of a surface patch $\sigma$, we have
$\gamma(t) = \sigma(u(t), v(t))$ and
$\dot{\gamma} = \dot{u} \sigma_u + \dot{v} \sigma_v$, so
\[
  \langle \dot{\gamma}, \dot{\gamma} \rangle =
  E \dot{u}^2 + 2 F \dot{u} \dot{v} + G \dot{v}^2,
\]
and the length of $\gamma$ is given by
\[
  \int (E \dot{u}^2 + 2 F \dot{u} \dot{v} + G \dot{v}^2)^{1/2}dt.
\]

\begin{example}
  For the plane
  \[
    \sigma(u, v) = a + up + vq  
  \]
  with $p$ and $q$ being perpendicular unit vectors, we have
  $\sigma_u = p, \sigma_v = q$, so
  $E = \|p\|^2 = 1,
  F = p \cdot q = 0,
  G = \|q\|^2 = 1$,
  and the first fundamental form is simply
  $du^2 + dv^2$.
\end{example}

\subsection{Isometries of surfaces}

\begin{defn}
  If $S_1$ and $S_2$ are surfaces, a smooth map $f: S_1 \to S_2$ is called
  a local isometry if it takes any curve in $S_1$ to a curve of the same
  length in $S_2$.
  If a local isometry $f: S_1 \to S_2$ exists,
  we say that $S_1$ and $S_2$ are locally isometric.
\end{defn}

\begin{remark}
  Every local isometry is a local diffeomorphism.
  Any composite of local isometries is a local isometry, and
  the inverse of any isometry is an isometry.
\end{remark}

\begin{theorem}
  A smooth map $f: S_1 \to S_2$ is a local isometry if and only if
  the symmetric bilinear forms $\langle\,,\,\rangle_p$ and
  $f^*\langle\,,\,\rangle_p$ on $T_p S_1$ are equal for all $p \in S_1$.
\end{theorem}

Thus, $f$ is a local isometry if and only if
\[
  \langle D_p f(v), D_p f(w) \rangle_{f(p)} =
  \langle v, w \rangle_p
\]
for all $p \in S_1$ and all $v, w \in T_p S_1$.

\begin{corollary}
  A local diffeomorphism $f: S_1 \to S_2$ is a local isometry
  if and only if, for any surface patch $\sigma_1$ and
  $f \circ \sigma_1$ of $S_1$ and $S_2$, respectively,
  have the same first fundamental form.
\end{corollary}

\begin{defn}
  A tangent developable is the union of the tangent lines to a curve in
  $\BR^3$.
  The tangent line to a curve $\gamma$ at a point $\gamma(u)$
  is the straight line passing through $\gamma(u)$ and parallel
  to the tangent vector $\dot{\gamma}(u)$.
\end{defn}

\begin{proposition}
  Any tangent developable is locally isometric to a plane.
\end{proposition}

\subsection{Conformal mappings of surfaces}

Suppose that two curves $\gamma$ and $\tilde \gamma$ on a surface $S$
intersect at a point $p$.
The angle $\theta$ of intersection of $\gamma$ and $\tilde \gamma$
at $p$ is defined to be the angle between the tangent vectors
$\dot{\gamma}$ and $\dot{\tilde \gamma}$. Using the dot product
formula for the angle between vectors, we see that $\theta$ is given by
\begin{align*}
  \cos \theta &= \frac{
    \dot{\gamma} \cdot \dot{\tilde \gamma}
  }{
    \|\dot{\gamma}\| \|\dot{\tilde \gamma}\|
  }\\
  &= \frac{
    E \dot{u} \dot{\tilde u} +
    F(\dot{u}\dot{\tilde v} +
    \dot{\tilde v}\dot{v}) + G \dot{v} \dot{\tilde v}
  }{
    (
      E \dot{u}^2 +
      2 F \dot{u}\dot{v} +
      G \dot{v}^2
    )^{1/2}
    (
      E \dot{\tilde u}^2 +
      2 F \dot{\tilde u}\dot{\tilde v} +
      G \dot{\tilde v}^2
    )^{1/2}.
  }
\end{align*}

\begin{defn}
  If $S_1$ and $S_2$ are surfaces, a conformal map $f: S_1 \to S_2$ is a
  local diffeomorphism such that, if $\gamma_1$ and $\tilde \gamma_1$
  are any two curves on $S_1$ that intersect, say at a point $p \in S_1$,
  and if $\gamma_2$ and $\tilde \gamma_2$ are their images under $f$,
  the angle of intersection of $\gamma_1$ and $\tilde \gamma_1$ at $p$
  is equal to the angle of intersection of $\gamma_2$ and $\tilde \gamma_2$
  at $f(p)$.
\end{defn}

In short, $f$ is conformal if and only if it preserves angles.

\begin{theorem}
  A local diffeomorphism $f: S_1 \to S_2$ is conformal if and only if
  there is a function $\lambda: S_1 \to \BR$ such that
  \[
    f^*\langle v, w \rangle_p  = \lambda(p) \langle v, w \rangle_p \text{ for all }
    p \in S_1 \text{ and } v, w \in T_p S_1.
  \]
\end{theorem}

\begin{corollary}
  A local diffeomorphism $f: S_1 \to S_2$ is conformal if and only if, for
  any surface patch $\sigma$ of $S_1$, the first fundamental forms of the
  patches $\sigma$ of $S_1$ and $f \circ \sigma$ of $S_2$ are proportional.
\end{corollary}

\begin{theorem}
  Every surface has an atlas consisting of conformal surface patches.
\end{theorem}