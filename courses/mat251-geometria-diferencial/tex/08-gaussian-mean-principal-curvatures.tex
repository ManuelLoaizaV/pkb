\section{Gaussian, mean and principal curvatures}

\subsection{Gaussian and mean curvatures}

\begin{defn}
  Let $\mathcal{W}$ be the Weingarten map of an oriented surface $S$
  at a point $p \in S$. The Gaussian curvature $K$ and mean curvature
  $H$ of $S$ at $p$ are defined by
  \[
    K = \det(\mathcal{W}), \, H = \frac{1}{2}\trace(\mathcal{W}).
  \]
\end{defn}

Define symmetric $2 \times 2$ matrices $\mathcal{F}_I$ and $\mathcal{F}_{II}$ by
\[
  \mathcal{F}_I = \begin{pmatrix}
    E & F \\
    F & G
  \end{pmatrix}, \,
  \mathcal{F}_{II} = \begin{pmatrix}
    L & M \\
    M & N
  \end{pmatrix}.
\]

\begin{proposition}
  Let $\sigma$ be a surface patch of an oriented surface $S$.
  Then, with the above notation, the matrix of $\mathcal{W}_{p, S}$
  with respect to the basis $\{\sigma_u, \sigma_v\}$ of $T_p S$ is
  $\mathcal{F}_I^{-1} \mathcal{F}_{II}$.
\end{proposition}

\begin{corollary}
  We have
  \[
    H = \frac{LG - 2MF + NE}{2(EG - F^2)}, \,
    K = \frac{LN - M^2}{EG - F^2}.
  \]
\end{corollary}

\subsection{Principal curvatures of a surface}

\begin{proposition}
  Let $p$ be a point of a surface $S$.
  There are scalars $\kappa_1, \kappa_2$ and a basis $\{t_1, t_2\}$
  of the tangent plane $T_p S$ such that
  \[
    \mathcal{W}(t_1) = \kappa_1 t_1, \,
    \mathcal{W}(t_2) = \kappa_2 t_2.
  \]
  Moreover, if $\kappa_1 \neq \kappa_2$, then $\langle t_1, t_2 \rangle = 0$.
\end{proposition}

\begin{corollary}
  If $p$ is a point of a surface $S$,
  there is an orthonormal basis of the tangent plane $T_p S$
  consisting of principal vectors.
\end{corollary}

\begin{proposition}
  If $\kappa_1$ and $\kappa_2$ are the principal curvatures of a surface,
  the mean and Gaussian curvatures are given by
  \[
    H = \frac{1}{2}(\kappa_1 + \kappa_2), \,
    K = \kappa_1 \kappa_2.
  \]
\end{proposition}

\begin{theorem}[Euler's Theorem]
  Let $\gamma$ be a curve on an oriented surface $S$,
  and let $\kappa_1$ and $\kappa_2$ be the principal curvatures of $\sigma$,
  with non-zero principal vectors $t_1$ and $t_2$.
  Then, the normal curvature of $\gamma$ is
  \[
    \kappa_n = \kappa_1 \cos^2 \theta + \kappa_2 \sin^2 \theta,  
  \]
  where $\theta$ is the oriented angle $\widehat{t_1 \dot{\gamma}}$.
\end{theorem}

\begin{corollary}
  The principal curvatures at a point of a surface are the maximum and minimum
  values of the normal curvature of all curves on the surface that pass through
  the point. Moreover, the principal vectores are the tangent vectors of the curves
  giving these maximum and minimum values.
\end{corollary}

\begin{proposition}
  The principal curvatures are the roots of the equation
  \[
    \begin{vmatrix}
      L - \kappa E & M - \kappa F \\
      M - \kappa F & N - \kappa G
    \end{vmatrix}
    = 0,
  \]
  and the principal vectors corresponding to the principal curvature $\kappa$
  are the tangent vectors $t = \xi \sigma_u + \eta \sigma_v$, such that
  \[
    \begin{pmatrix}
      L - \kappa E & M - \kappa F \\
      M - \kappa F & N - \kappa G
    \end{pmatrix}
    \begin{pmatrix}
      \xi \\
      \eta
    \end{pmatrix}
    =
    \begin{pmatrix}
      0 \\
      0
    \end{pmatrix}
    .
  \]
\end{proposition}

\begin{proposition}
  Let $S$ be a (connected) surface of which every point is umbilic.
  Then, $S$ is an open subset of a plane or a sphere.
\end{proposition}

\subsection{Gaussian curvature of compact surfaces}

The relative signs of the principal curvatures at a point $p$ of
a surface $S$ determine the shape of $S$ near $p$.
In fact, since the Gaussian curvature $K$ of $S$ is the product
of its principal curvatures, then

\begin{itemize}
  \item If $K > 0$ at $p$, then $p$ is an elliptic point.
  \item If $K < 0$ at $p$, then $p$ is a hyperbolic point.
  \item If $K = 0$ at $p$, then $p$ is either a parabolic point or a planar point.
\end{itemize}

\begin{proposition}
  If $S$ is a compact surface, there is a point of $S$ at which its
  Gaussian curvature $K$ is greater than $0$.
\end{proposition}