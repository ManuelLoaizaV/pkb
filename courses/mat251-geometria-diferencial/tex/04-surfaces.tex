\section{Surfaces}

\begin{defn}
  A subset $S$ of $\BR^3$ is a surface if, for every point $p \in S$
  there is an open set $U$ in $\BR^2$ and an open set $W$ in $\BR^3$
  containing $p$ such that $S \cap W$ is homeomorphic to $U$.
  A subset of a surface $S$ of the form $S \cap W$, where $W$
  is an open subset of $\BR^3$, is called an open subset of $S$.
  A homeomorphism $\sigma: U \to S \cap W$ as in this definition
  is called a surface patch or parametrization of
  the open subset $S \cap W$ of $S$.
  A collection of such surface patches whose images cover the
  whole of $S$ is called an atlas of $S$.
\end{defn}

\begin{example}
  Every plane in $\BR^3$ is a surface with an atlas consisting of a single surface patch.
\end{example}

\begin{example}
  A circular cylinder is the set of points of $\BR^3$ that are at a fixed distance
  (the radius of the cylinder) from a fixed straight line (its axis).
  For instance, the circular cylinder of radius $1$ and axis the $z$-axis is
  \[
    S = \{(x, y, z) \in \BR^3 : x^2 + y^2 = 1\}/
  \]
  The simplest parametrization of $S$ is 
  \[
    \sigma(u, v) = (\cos u, \sin u, v).
  \]
\end{example}

% TODO: Add more examples

\subsection{Smooth surfaces}

\begin{defn}
  A surface patch $\gamma: U \to \BR^3$ is called regular if
  it is smooth and the vectors $\gamma_u$ and $\gamma_v$ are
  linearly independent at all points $(u, v) in U$.
  Equivalently, $\gamma$ should be smooth and the vector
  product $\gamma_u \times \gamma_v$ should be non-zero at every point of $U$.
\end{defn}

\begin{defn}
  If $S$ is a surface, and allowable surface patch for $S$ is a regular
  surface patch $\gamma: U \to \BR^3$ such that $\gamma$ is a
  homeomorphism from $U$ to an open subset of $S$.
  A smooth surface is a surface $S$ such that, for any point $p \in S$,
  there is an allowable surface patch $\gamma$ as above such that $p \in \gamma(U)$.
  A collection $A$ of allowable surface patches for a surface $S$ such that
  every point of $S$ is in the image of at least one patch in $A$
  is called an atlas for the smooth surface $S$.
\end{defn}

\begin{proposition}
  The transition maps of a smooth surface are smooth.
\end{proposition}

\subsection{Tangents and derivatives}

\begin{defn}
  A tangent vector to a surface $S$ at a point $p \in S$ is the tangent
  vector at $p$ of a curve in $S$ passing through $p$.
  The tangent space $T_p S$ of $S$ at $p$ is the set of all tangent vectors
  to $S$ at $p$.
\end{defn}

\begin{proposition}
  Let $\gamma: U \to \BR^3$ be a patch of a surface $S$ containing a point
  $p \in S$, and let $(u, v)$ be coordinates in $U$.
  The tangent space to $S$ at $p$ is the vector subspace of $\BR^3$
  spanned by the vectors $\gamma_u$ and $\gamma_v$.
\end{proposition}

\begin{defn}
  With the above notation, the derivative $D_p f$ of $f$ at the point $p \in S$
  is the map $D_p f: T_p S \to T_{f(p)} \tilde S$ such that $D_p f(w) = \tilde w$
  for any tangent vector $w \in T_p S$.
\end{defn}

\begin{proposition}
  If $f: S \to \tilde S$ is a smooth map between surfaces and $p \in S$,
  the derivative $D_p f: T_p s \to T_{f(p)} \tilde S$ is a linear map.
\end{proposition}

Let $\gamma: U \to \BR^3$ be a surface patch of $S$ containing $p$,
say $p = \gamma(u_0, v_0)$, and let $\alpha, \beta$ be the smooth
functions on $U$ such that
\[
  f(\gamma(u, v)) = \tilde \gamma(\alpha(u, v), \beta(u, v)).
\]
Then,
\[
  D_p f(w) = (\lambda \alpha_u + \mu \beta_v) \tilde \sigma_{\tilde u}   + (\lambda \beta_u + \mu \beta_v) \tilde \sigma_{\tilde v}.
\]
The matrix of the linear map $D_p f$ with respect to the basis $\{\sigma_u, \sigma_v\}$
of $T_p f$ and the basis $\{\tilde \sigma_{\tilde u}, \tilde \sigma_{\tilde v}\}$ of
$T_{f(p)} \tilde S$ is the Jacobian matrix
\[
  \begin{pmatrix}
    \alpha_u & \alpha_v \\
    \beta_u & \beta_v
  \end{pmatrix}
\]
of the smooth map $(u, v) \mapsto (\alpha(u, v), \beta(u, v))$.

\begin{proposition}
  \begin{itemize}
    \item If $S$ is a surface and $p \in S$, the derivative at $p$ of the
    identity map $S \to S$ is the identity map $T_p S \to T_p S$.
    \item If $S_1, S_2$ and $S_3$ are surfaces and $f_1 : S_1 \to S_2$
    and $f_2: S_2 \to S_3$ are smooth maps, then for all $p \in S_1$,
    \[
      D_p(f_2 \circ f_1) = D_{f_1(p)} f_2 \circ D_p f_1.
    \]
    \item If $f: S_1 \to S_2$ is a diffeomorphism, then for all $p \in S_1$
    the linear map $D_p f: T_p S_1 \to T_{f(p)} S_2$ is invertible.
  \end{itemize}
\end{proposition}

\begin{proposition}
  Let $S$ and $\tilde S$ be surfaces and let $f: S \to \tilde S$ be a
  smooth map. Then, $f$ is a local diffeomorphism if and only if,
  for all $p \in S$, the linear map $D_p f: T_p S \to T_{f(p)} \tilde S$
  is invertible.
\end{proposition}

\subsection{Normals and orientability}

Since the tangent plane $T_p S$ of a surface $S$ at a point $p \in S$
passes through the origin of $\BR^3$, it is completely determined
by giving a unit vector perpendicular to it, called a unit normal to
$S$ at $p$. We choose one of two such vectors
\[
  N_{\sigma}   = \frac{\sigma_u \times \sigma_v}{\|\sigma_u \times \sigma_v\|}
\]
called the standard unit normal of the surface patch $\sigma$ at $p$.

\begin{defn}
  A surface $S$ is orientable if there exists an atlas $A$ for $S$ with the
  property that, if $\Phi$ is the transition map between any two surface patches
  in $A$, then $\det(J(\Phi)) > 0$ where $\Phi$ is defined.
\end{defn}

\begin{proposition}
  Let $S$ be an orientable surface equipped with an atlas $A$.
  Then, there is a smooth choice of unit normal at any point of $S$:
  take the standard unit normal of any surface patch in $A$.
\end{proposition}

An oriented surface is a surface $S$ together with a smooth choice
of unit normal $N$ at each point, i.e., a smooth map $N: S \to \BR^3$
such that, for all $p \in S, N(p)$ is a unit vector perpendicular to
$T_p S$.