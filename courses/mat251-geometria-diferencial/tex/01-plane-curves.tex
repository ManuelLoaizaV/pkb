\section{Curves in the plane and in space}

\subsection{What is a curve?}

\begin{defn}
  A parametrized curve in $\BR^n$
  is a map $\gamma: (\alpha, \beta) \to \BR^n$,
  for some $\alpha, \beta$ with $-\infty \leq \alpha < \beta \leq \infty$.
\end{defn}

\begin{defn}
  If $\gamma$ is a parametrized curve, its first derivative $\dot{\gamma}(t)$
  is called the tangent vector of $\gamma$ at the point $\gamma(t)$.
\end{defn}

\begin{proposition}
  If $\dot{\gamma}(t) = a$ for all $t$, where $a$ is a constant vector,
  we have, integrating componentwise,
  \[
    \gamma(t) = \int \frac{d\gamma}{dt} dt = \int a \, dt = ta + b,
  \]
  where $b$ is another constant vector.
  If $a \neq 0$, this is the parametric equation of the straight
  line parallel to $a$ and passing through the point $b$.
  If $a = 0$, the image of $\gamma$ is a single point (namely, $b$).
\end{proposition}

\subsection{Arc-length}

We recall that, if $v = (v_1, \dots, v_n)$ is a vector in $\BR^n$, its length is
\[
    \|v\| = \sqrt{v_1^2 + \cdots + v_n^2}.
\]

\begin{defn}
  The arc-length of a curve $\gamma$ starting at the point $\gamma(t_0)$
  is the function $s(t)$ given by
  \[
    s(t) = \int_{t_0}^t \|\dot{\gamma}(u)\| \, du.
  \]
\end{defn}

\begin{defn}
  If $\gamma: (\alpha, \beta) \to \BR^n$ is a parametrized curve,
  its speed at the point $\gamma(t)$ is $\|\dot{\gamma}(t)\|$,
  and $\gamma$ is said to be unit-speed curve if $\dot{\gamma}(t)$
  us unit vector for all $t \in (\alpha, \beta)$.
\end{defn}

We recall that the dot product of vectors
$a = (a_1, \dots, a_n)$ and
$b = (b_1, \dots, b_n)$ in $\BR^n$ is
\[
  a \cdot b = \sum_{i = 1}^n a_i b_i.
\]

\begin{proposition}
  Let $n(t)$ be a unit vector that is a smooth function of a parameter $t$.
  Then, the dot product
  \[
    \dot{n} \cdot n(t) = 0  
  \]
  for all $t$. In particular, if $\gamma$ is a unit-speed curve,
  then $\ddot{\gamma}$ is zero or perpendicular to $\dot{\gamma}$.
\end{proposition}

\subsection{Reparametrization}

\begin{defn}
  A parametrized curve $\tilde \gamma: (\tilde \alpha, \tilde \beta) \to \BR^n$
  is a reparametrization of a parametrized curve $\gamma: (\alpha, \beta) \to \BR^n$
  if there is a smooth bijective map $\phi: (\tilde \alpha, \tilde \beta) \to (\alpha, \beta)$
  (the reparametrization map) such that the inverse map $\phi^{-1}: (\alpha, \beta) \to (\tilde \alpha, \tilde \beta)$
  is also smooth and
  \[
    \tilde \gamma(\tilde t) = \gamma(\phi(\tilde t)) \text{ for all } \tilde t \in (\tilde \alpha, \tilde \beta).
  \]
\end{defn}

\begin{note}
  Since $\phi$ has a smooth inverse, $\gamma$ is a reparametrization of $\tilde \gamma$:
  \[
    \tilde \gamma(\phi^{-1}(t)) = \gamma(\phi(\phi^{-1}(t))) = \gamma(t) \text{ for all } t \in (\alpha, \beta).
  \]
  Two curves that are reparametrizations of each other have the same image, so they should have the same geometric properties
\end{note}

\begin{defn}
  A point $\gamma(t)$ of a parametrized curve $\gamma$ is called a regular point
  if $\dot(\gamma)(t) \neq 0$; otherwise $\gamma(t)$ is a singular point of $\gamma$.
  A curve is regular if all of its points are regular.
\end{defn}

\begin{proposition}
  Any reparametrization of a regular curve is regular.
\end{proposition}

\begin{proposition}
  If $\gamma(t)$ is a regular curve, its arc-length $s$,
  starting at any point of $\gamma$, is a smooth function of $t$.
\end{proposition}

\begin{proposition}
  A parametrized curve has a unit-speed reparametrization if and only if it is regular.
\end{proposition}

\begin{corollary}
  Let $\gamma$ be a regular curve and let $\tilde \gamma$ be a unit-speed reparametrization of $\gamma$:
  \[
    \tilde \gamma(u(t)) = \gamma(t) \text{ for all } t,
  \]
  where $u$ is a smooth function of $t$.
  Then, if $s$ is the arc-length of $\gamma$ (starting at any point), we have
  \[
    u = \pm s + c,
  \]
  where $c$ is a constant.
  Conversely, if $u$ is given by the last equation for some value of $c$ and with either sign,
  then $\tilde \gamma$ is a unit-speed reparametrization of $\gamma$.
\end{corollary}

% TODO:
% Add Closed Curves subsection
% Add Level Curves versus Parametrized Curves subsection