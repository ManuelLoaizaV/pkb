\section{Curves in the plane and in space}

\subsection{What is a curve?}

\begin{defn}
  A parametrized curve in $\BR^n$
  is a map $\gamma: (\alpha, \beta) \to \BR^n$,
  for some $\alpha, \beta$ with $-\infty \leq \alpha < \beta \leq \infty$.
\end{defn}

\begin{defn}
  If $\gamma$ is a parametrized curve, its first derivative $\dot{\gamma}(t)$
  is called the tangent vector of $\gamma$ at the point $\gamma(t)$.
\end{defn}

\begin{proposition}
  If $\dot{\gamma}(t) = a$ for all $t$, where $a$ is a constant vector,
  we have, integrating componentwise,
  \[
    \gamma(t) = \int \frac{d\gamma}{dt} dt = \int a \, dt = ta + b,
  \]
  where $b$ is another constant vector.
  If $a \neq 0$, this is the parametric equation of the straight
  line parallel to $a$ and passing through the point $b$.
  If $a = 0$, the image of $\gamma$ is a single point (namely, $b$).
\end{proposition}

\subsection{Arc-lenght}

We recall that, if $v = (v_1, \dots, v_n)$ is a vector in $\BR^n$, its lenght is
\[
    \|v\| = \sqrt{v_1^2 + \cdots + v_n^2}.
\]

\begin{defn}
  The arc-length of a curve $\gamma$ starting at the point $\gamma(t_0)$
  is the function $s(t)$ given by
  \[
    s(t) = \int_{t_0}^t \|\dot{\gamma}(u)\| \, du.
  \]
\end{defn}

\begin{defn}
  If $\gamma: (\alpha, \beta) \to \BR^n$ is a parametrized curve,
  its speed at the point $\gamma(t)$ is $\|\dot{\gamma}(t)\|$,
  and $\gamma$ is said to be unit-speed curve if $\dot{\gamma}(t)$
  us unit vector for all $t \in (\alpha, \beta)$.
\end{defn}

We recall that the dot product of vectors
$a = (a_1, \dots, a_n)$ and
$b = (b_1, \dots, b_n)$ in $\BR^n$ is
\[
  a \cdot b = \sum_{i = 1}^n a_i b_i.
\]

\begin{proposition}
  Let $n(t)$ be a unit vector that is a smooth function of a parameter $t$.
  Then, the dot product
  \[
    \dot{n} \cdot n(t) = 0  
  \]
  for all $t$. In particular, if $\gamma$ is a unit-speed curve,
  then $\ddot{\gamma}$ is zero or perpendicular to $\dot{\gamma}$.
\end{proposition}