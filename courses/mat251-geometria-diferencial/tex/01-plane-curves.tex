\section{Curves in the plane and in space}

\subsection{What is a curve?}

\begin{defn}
  A parametrized curve in $\BR^n$
  is a map $\gamma: (\alpha, \beta) \to \BR^n$,
  for some $\alpha, \beta$ with $-\infty \leq \alpha < \beta \leq \infty$.
\end{defn}

\begin{defn}
  If $\gamma$ is a parametrized curve, its first derivative $\dot{\gamma}(t)$
  is called the tangent vector of $\gamma$ at the point $\gamma(t)$.
\end{defn}

\begin{proposition}
  If $\dot{\gamma}(t) = a$ for all $t$, where $a$ is a constant vector,
  we have, integrating componentwise,
  \[
    \gamma(t) = \int \frac{d\gamma}{dt} dt = \int a \, dt = ta + b,
  \]
  where $b$ is another constant vector.
  If $a \neq 0$, this is the parametric equation of the straight
  line parallel to $a$ and passing through the point $b$.
  If $a = 0$, the image of $\gamma$ is a single point (namely, $b$).
\end{proposition}