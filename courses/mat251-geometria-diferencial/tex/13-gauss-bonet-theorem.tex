\section{The Gauss-Bonnet theorem}

\subsection{Gauss-Bonnet for simple closed curves}

\begin{defn}
  A curve $\gamma(t) = \sigma(u(t), v(t))$ on a surface patch $\sigma: U \to \BR^3$ is called
  a simple closed curve with period $T$ if $\pi(t) = (u(t), v(t))$ is a simple closed curve in
  $\BR^2$ with period $T$ such that region int$(\pi)$ of $\BR^2$ enclosed by $\pi$ is
  entirely contained in $U$.
  The curve $\gamma$ is said to be positively-oriented if $\pi$ is positively oriented.
  Finally, the image of int$(\pi)$ under the map $\sigma$ is defined to be the interior of $\gamma$.
\end{defn}

\begin{theorem}
  Let $\gamma(s)$ be a unit-speed simple closed curve on a surface patch $\sigma$ of length $l(\gamma)$,
  and assume that $\gamma$ is positively-oriented. Then,
  \[
    \int_0^{l(\gamma)} \kappa_g \, ds = 2\pi - \int_{\text{int}(\gamma)} K dA_{\sigma},
  \]
  where $\kappa_g$ is the geodesic curvature of $\gamma$,
  $K$ is the Gaussian curvature of $\sigma$ and
  $dA_{\sigma}$ is the area element of $\sigma$.
\end{theorem}

We shall make use of a smooth orthonormal basis $\{e', e''\}$ of the tangent plane at each point of the surface patch.

\begin{lemma}
  With the above notation, we have
  \[
    e'_u \cdot  e''_v - e''_u \cdot e'_v = \frac{LN - M^2}{(EG - F^2)^{1 / 2}}.
  \]
  and
  \[
    \int_0^{l(\gamma)} \dot{\theta} ds = 2\pi.
  \]
\end{lemma}

\subsection{Gauss-Bonnet for curvilinear polygons}

\begin{defn}
  A curvilinear polygon in $\BR^2$ is a continuous map $\pi: \BR \to \BR^2$ such that, for
  some real number $T$ and some points $0 = t_0 < t_1 < \cdots < t_n = T$:
  \begin{itemize}
    \item $\pi(t) = \pi(t')$ if and only if $t' - t$ is an integer multiple of $T$.
    \item $\pi$ is smooth on each of the open intervals $(t_0, t_1), \dots, (t_{n - 1}, t_n)$.
    \item The one-sided derivative exist for $i = 1, \dots, n$ and are non-zero and not parallel.
  \end{itemize}
  The points if $\gamma(t_i)$ for $i = 1, \dots, n$ are called the vertices of the curvilinear
  polygon $\pi$, and the segments of it corresopnding to the open intervals $(t_{i - 1}, t_i)$
  are called its edges.
\end{defn}

\begin{theorem}
  Let $\gamma$ be a positively-oriented unit-speed curvilinear polygon with $n$ edges on a
  surface $\sigma$, and let $\alpha_1, \alpha_2, \dots, \alpha_n$ be the interior angles at its vertices.
  Then,
  \[
    \int_0^{l(\gamma)} \kappa_g \, ds = \sum_{i = 1}^n \alpha_i - (n - 2) \pi - \int_{\text{int}(\gamma)} K \, dA_{\sigma}.
  \]
\end{theorem}

\begin{corollary}
  If $\gamma$ is a curvilinear polygon with $n$ edges each of which is an arc of a geodesic,
  then the internal angles $\alpha_1, \alpha_2, \dots, \alpha_n$ of the polygon satisfy the equation
  \[
    \sum_{i = 1}^n \alpha_i = (n - 2) \pi + \int_{\text{int}(\gamma)} K \, dA_{\sigma}.
  \]
\end{corollary}

\subsection{Gauss-Bonnet for compact surfaces}

\begin{defn}
  Let $S$ be a surface, with atlas consisting of the patches $\sigma_i: U_i \to \BR^3$.
  A triangulation of $S$ is a collection of curvilinear polygons, each of which is contained,
  together with its interior, in one of the $\sigma_i(U_i)$, such that:
  \begin{itemize}
    \item Every point of $S$ is in at least one of the curvilinear polygons.
    \item Two curvilinear polygons are either disjoint, or their intersection is a common edge or a common vertex.
    \item Each edge is an edge of exactly two polygons.
  \end{itemize}
\end{defn}

\begin{theorem}
  Every compact surface has a triangulation with finitely many polygons.
\end{theorem}

\begin{defn}
  The Euler Number $\chi$ of a triangulation of a compact surface $S$ with 
  finitely many polygons is
  \[
    \chi = V - E + F,  
  \]
  where
  \begin{align*}
    V &= \text{the total number of vertices of the triangulation},\\
    E &= \text{the total number of edges of the triangulation},\\
    F &= \text{the total number of polygons of the triangulation}.
  \end{align*}
\end{defn}

\begin{theorem}
  Let $S$ be a compact surface. Then, for any triangulation of $S$,
  \[
    \int_S K dA = 2\pi\chi,
  \]
  where $\chi$ is the Euler number of the triangulation.
\end{theorem}

\begin{corollary}
  The Euler number $\chi$ of a triangulation of a compact surface $S$ depends
  only on $S$ and not on the choice of triangulation.
\end{corollary}

\begin{theorem}
  The Euler number of the compact surface $T_g$ of genus $g$ is $2 - 2g$.
\end{theorem}

\begin{corollary}
  We have
  \[
    \int_{T_g} K dA = 4\pi(1 - g).
  \]
\end{corollary}

\subsection{Map colouring}

\noindent\textbf{Map Colouring Problem} For a given compact surface $S$, what is the smallest
positive integer $n$ such that every map on $S$ can be $n$-coloured?

This smallest integer $n$ is called the chromatic number of $S$.

\noindent\textbf{The Four Colour Conjecture} The chromatic number of a sphere is $4$.

Let $h(\chi)$ be the largest integet $\leq N(\chi)$.

\noindent\textbf{Heawood's Conjecture} The chromatic number of a compact surface of Euler number
$\chi \leq 0$ is $h(\chi)$.

\begin{theorem}
  Any compact surface of Euler number $\chi \leq 0$ can be $h(\chi)$-coloured.
\end{theorem}

\begin{theorem}
  The chromatic number of a torus is $7$.
\end{theorem}

\begin{proposition}[Five Neighbours Theorem]
  Ever map on a sphere has at least one country with five or fewer neighbours.
\end{proposition}

\begin{corollary}[Six Colour Theorem]
  Every map on a sphere can be six-coloured.
\end{corollary}

\subsection{Holonomy and Gaussian curvature}

\begin{proposition}
  Let $\gamma$ be a unit-speed curve on a surface patch $\sigma$ and let $\bm{v}$ be a non-zero
  parallel vector field along $\gamma$. Let $\varphi$ be the oriented angle $\widehat{\dot{\gamma}\bm{v}}$
  from $\dot{\gamma}$ to $\bm{v}$. Then, the geodesic curvature of $\gamma$ is
  \[
    \kappa_g = -\frac{d\varphi}{ds}.
  \]
\end{proposition}

\begin{proposition}
  Let $\gamma$ be a positively-oriented unit-speed simple closed curve on a surface $\sigma$,
  let $\kappa_g$ be the geodesic curvature of $\gamma$, and let $v$ be a non-zero parallel vector
  field along $\gamma$. Then, on going once around $\gamma$, $v$ rotates through an angle
  \[
    2\pi - \int_0^{l(\gamma)} \kappa_g ds.
  \]
\end{proposition}

\begin{defn}
  If $\gamma$ is a unit-speed closed curve on a surface $S$, the angle of the previous proposition
  is called the holonomy around $\gamma$, and is denoted by $h_{\gamma}$.
\end{defn}

\begin{theorem}
  Let $\gamma$ be positively-oriented simple closed curve on a surface patch $\sigma$,
  let $h_{\gamma}$ be the holonomy around $\gamma$,
  and let $K$ be the Gaussian curvature of $\sigma$.
  Then,
  \[
    h_{\gamma}   = \int_{\text{int}(\gamma)} K dA_{\gamma}.
  \]
\end{theorem}

\begin{proposition}
  Suppose that a surface $S$ has the property that, for any two points $p, q \in S$,
  the parallel transport $\Pi_{\gamma}^{pq}$ is independent of the curve $\gamma$ joining $p$ and $q$.
  Then $S$ is flat.
\end{proposition}

