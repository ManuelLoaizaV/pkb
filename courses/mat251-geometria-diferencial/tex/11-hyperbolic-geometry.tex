\section{Hyperbolic geometry}

\subsection{Upper half-plane model}

The pseudosphere is parametrized as
\[
  \tilde{\sigma}(v, w) = \left(
    \frac{1}{w}\cos v,
    \frac{1}{w} \sin v,
    \sqrt{1 - \frac{1}{w^2}} - \cosh^{-1}w
  \right).
\]
with first fundamental form
\[
  \frac{dv^2 + dw^2}{w^2}.
\]

\begin{proposition}
  Hyperbolic angles in $\mathcal{H}$ are the same as Euclidean angles.
\end{proposition}

\begin{proposition}
  The geodesics in $\mathcal{H}$ are the half-lines parallel to the imaginary axis
  and the semicircles with centres on the real axis.
\end{proposition}

\begin{proposition}
  \begin{itemize}
    \item There is a unique hyperbolic line passing through any tow distinct points of $\mathcal{H}$.
    \item The parallel axiom does not hold in $\mathcal{H}$.
  \end{itemize}
\end{proposition}

\begin{proposition}
  THe hyperbolic distance between two points $a, b \in \mathcal{H}$ is
  \[
    d_{\mathcal{H}}(a, b) = 2 \tanh^{-1}\frac{|b - a|}{|b - \overline{a}|}.
  \]
\end{proposition}

\begin{theorem}
  Let $\mathcal{P}$ be a $n$-sided hyperbolic polygon in $\mathcal{H}$ with internal angles
  $\alpha_1, \alpha_2, \dots, \alpha_n$.
  Then, the hyperbolic area of the polygon is given by
  \[
    A(\mathcal{P}) = (n - 2)\pi - \alpha_1 - \dots \alpha_n.
  \]
\end{theorem}

\begin{lemma}
  Let $a$ and $b$ be the endpoints of a segment $l$ of a hyperbolic line in $\mathcal{H}$
  that forms part of a semicircle with centre $p$ on the real axis, and suppose that the
  radius vectors joining $p$ to $a$ and $p$ to $b$ make angles $\varphi$ and $\psi$,
  respectively, with the positive real axis. Then,
  \[
    \int_l \frac{dv}{w}   = \varphi - \psi.
  \]
\end{lemma}

\subsection{Isometries of $\mathcal{H}$}

It is easy to define some isometries of $\mathcal{H}$:

\begin{itemize}
  \item Translations parallel to the real axis, given by
  \[
    T_a(z) = z + a, a \in \BR.  
  \]
  \item Reflections in lines parallel to the imaginary axis, given by
  \[
    R_a(z) = 2a - \overline{z}, a \in \BR.
  \]
  \item Dilations by a factor $a > 0$, given by
  \[
    D_a(z) = az.  
  \]
  \item Inversions in circles with centres on th e real axis.
  The inversion in the circle with centre $a \in \BR$ and radius $r > 0$ is
  \[
    \mathcal{I}_{a, r}(z) = a + \frac{r^2}{\overline{z} - a}.
  \]
\end{itemize}

\begin{proposition}
  Any composite of a finite number of maps of the types defined above is an isometry of $\mathcal{H}$.
\end{proposition}

\begin{proposition}
  The inversion $\mathcal{I}_{a, r}$ in the circle with centre $a \in \BR$ and radius $r > 0$
  takes hyperbolic lines that intersect the real axis perpendicularly at $a$ to half-lines,
  and all other hyperbolic lines to semicircles.
\end{proposition}

\begin{proposition}
  Let $l_1$ and $l_2$ be hyperbolic lines in $\mathcal{H}$, and let $z_1$ and $z_2$
  be points on $l_1$ and $l_2$, respectively.
  Then, there is an isometry of $\mathcal{H}$ that takes $l_1$ to $l_2$ and $z_1$ to $z_2$.
\end{proposition}

\begin{theorem}
  In hyperbolic geometry, similar triangles are congruent.
\end{theorem}

\subsection{Poincar\'e disc model}

We consider the transformation
\[
  \mathcal{P}(z) = \frac{z - i}{z + i}.
\]
It defines a bijection between the complex plane with the point $-i$ removed
and the complex plane with the point $1$ removed, its inverse being
\[
  \mathcal{P}^{-1}(z) = \frac{z + 1}{i(z - 1)}.
\]

Let the unit disc be $\mathcal{D} = \{z \in \BC : |z| < 1\}$.

\begin{defn}
  The Poincar\'e disc model $\mathcal{D}_P$ of hyperbolic geometry is the disc $\mathcal{D}$
  equipped with the first fundamental form for which $\mathcal{P}: \mathcal{H} \to \mathcal{D}_P$
  is an isometry.
\end{defn}

\begin{proposition}
  The first fundamental form of $\mathcal{D}_P$ is
  \[
    \frac{4(dv^2 + dw^2)}{(1 - v^2 - w^2)^2}.
  \]
  In particular, $\mathcal{D}_P$ is a conformal model of hyperbolic geometry.
\end{proposition}

\begin{proposition}
  \begin{itemize}
    \item Let $\Gamma$ be a circle that intersects $\mathcal{C}$ perpendicularly.
    Then, inversion in $\Gamma$ is an isometry of $\mathcal{D}_P$.
    \item Let $l$ be a line passing through the origin (and so perpendicular to $\mathcal{C}$).
    Then, (Euclidean) reflection in $l$ is an isometry of $\mathcal{D}_P$.
  \end{itemize}
\end{proposition}

\begin{proposition}
  For $a, b \in \mathcal{D}_P$, we have
  \[
    d_{\mathcal{D}_P}(a, b) = 2\tanh^{-1}\frac{|b - a|}{|1 - \overline{a}b|}.
  \]
\end{proposition}

\begin{proposition}
  The hyperbolic lines in $\mathcal{D}_P$ are the lines and circles that intersect
  $\mathcal{C}$ perpendicularly.
\end{proposition}

\begin{theorem}
  Consider a hyperbolic triangle with angles $\alpha, \beta, \gamma$ and sides of length
  $A, B, C$. Then,
  \[
    \cosh C = \cosh A \cosh B - \sinh A \sinh B \cos \gamma.  
  \]
  This formula is called the hyperbolic cosine rule.
\end{theorem}

\begin{corollary}
  Suppose that a hyperbolic triangle has sides of lengths $A, B$ and $C$ and that
  the angle opposite the side of length $C$ is a right angle. Then,
  \[
    \cosh C = \cosh A \cosh B.  
  \]
\end{corollary}