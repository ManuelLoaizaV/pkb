\section{Curvature of surfaces}

\subsection{The second fundamental form}

Suppose that $\sigma$ is a surface patch in $\BR^3$ with standard unit
normal $\bm{N}$.
Writing
\[
  L = \sigma_{uu} \cdot \bm{N},
  M = \sigma_{uv} \cdot \bm{N},
  N = \sigma_{vv} \cdot \bm{N}.
\]
One calls the expression
\[
  L du^2 + 2 M du dv + N dv^2  
\]
the second fundamental form of the surface of patch $\sigma$.

\begin{example}
  Consider the plane
  \[
    \sigma(u, v) = a + up + vq.  
  \]
  Since $\sigma_u = p$ and $\sigma_v = q$, we have
  $\sigma_{uu} = \sigma_{uv} = \sigma_{uv} = 0$.
  Hence, the second fundamental form of a plane is zero.
\end{example}

\subsection{The Gauss and Weingarten maps}

Consider $\bm{N}$ the unit normal of an oriented surface $S$.
The values of $\bm{N}$ at the points of $S$ are recorded by its
Gauss map $\mathcal{G}_S$. This is the map from $S$ to the unit
sphere $S^2$ that assigns to any point $p \in S$ the point
$\bm{N}_p \in S^2$, where $\bm{N}_p$ is the unit normal of
$S$ at $p$.

\begin{defn}
  Let $p$ be a point of a surface $S$. The Weingarten map
  $\mathcal{W}_{p, S}$ of $S$ at $p$ is defined by
  \[
    \mathcal{W}_{p, S} = -D_p \mathcal{G}.
  \]
  The second fundamental form of $S$ at $p \in S$ is the bilinear
  form on $T_p S$ given by
  \[
    \langle v, w \rangle_{p, S} = 
    \langle \mathcal{W}_{p, S}(v), w \rangle_{p, s},
    v, w \in T_p S.
  \]
\end{defn}

\begin{proposition}
  Let $p$ be a point of a surface $S$, let $\sigma(u, v)$ be a surface
  patch of $S$ with $p$ in its image, and let
  $L du^2 + 2 M du dv + N dv^2$ be the second fundamental form of
  $\sigma$. Then, for any $\bm{v}, \bm{w} \in T_p S$,
  \[
    \langle \bm{v}, \bm{w} \rangle = 
    L du(\bm{v}) du(\bm{w}) +
    M(du(\bm{v}) dv(\bm{w}) + du(\bm{w}) dv(\bm{v})) +
    N dv(\bm{v}) dv(\bm{w}).
  \]
\end{proposition}

\begin{lemma}
  Let $\sigma(u, v)$ be a surface patch with standard unit normal
  $\bm{N}(u, v)$. Then,
  \[
    \bm{N}_u \cdot \sigma_u = -L,\,
    \bm{N}_u \cdot \sigma_v = \bm{N}_v \cdot  \sigma_u = -M,\,
    \bm{N}_v \cdot \sigma_v = -N.
  \]
\end{lemma}

\begin{corollary}
  The second fundamental form is a symmetric bilinear form.
  Equivalently, the Weingarten map is self-adjoint.
\end{corollary}

\subsection{Normal and geodesic curvatures}

If $\gamma$ is a unit-speed curve on an oriented surface $S$, then
\[
  \ddot{\gamma} = \kappa_n \bm{N} + \kappa_g \bm{N} \times \dot{\gamma}.
\]

\begin{defn}
  The scalars $\kappa_n$ and $\kappa_g$ are called the normal curvature
  and the geodesic curvature of $\gamma$, respectively.
\end{defn}

\begin{proposition}
  With the above notation, we have
  \begin{gather*}
    \kappa_n = \ddot{\gamma} \cdot \bm{N},\\
    \kappa_g = \ddot{\gamma} \cdot (\bm{N} \times \dot{\gamma}),\\
    \kappa^2 = \kappa_n^2 + \kappa_g^2,
    \kappa_n = \kappa \cos \psi,
    \kappa_g = \pm \kappa \sin \psi,
  \end{gather*}
  where $\kappa$ is the curvature of $\gamma$ and
  $\psi$ is the angle between $\bm{N}$ and the principal normal
  $\bm{n}$ of $\gamma$.
\end{proposition}

\begin{proposition}
  If $\gamma$ is a unit-speed curve on an oriented surface $S$,
  its normal curvature is given by
  \[
    \kappa_n = \langle \dot{\gamma}, \dot{\gamma} \rangle.
  \]
  If $\sigma$ is a surface patch of $S$ and $\gamma(t) = \sigma(u(t), v(t))$
  is a curve in $\sigma$, then
  \[
    \kappa_n = L \dot{u}^2 + 2 M \dot{u}  \dot{v}+ N \dot{v}^2.
  \]
\end{proposition}

\begin{proposition}[Meusnier's Theorem]
  Let $p$ be a point of a surface $S$ and let $v$ be a unit tangent
  vector to $S$ at $p$. Let $\Pi_{\theta}$ be the plane containing
  the line through $p$ parallel to $v$ and making an angle $\theta$ with
  the tangent plane $T_p S$, and assume that $\Pi_{\theta}$ is not
  parallel to $T_p S$. Suppose that $\Pi_{\theta}$ intersects $S$
  in a cruve with curvature $\kappa_{\theta}$. Then,
  $\kappa_{\theta} \sin \theta$ is independent of $\theta$.
\end{proposition}

\begin{corollary}
  The curvature $\kappa$, normal curvature $\kappa_n$ and
  geodesic curvature $\kappa_g$ of a normal section of a surface are
  related by
  \[
    \kappa_n = \pm \kappa, \, \kappa_g = 0.  
  \]
\end{corollary}

\subsection{Parallel transport and convariant derivative}

\begin{defn}
  Let $\gamma$ be a curve on a surface $S$ and let $v$ be a tangent
  vector field along $\gamma$. The covariant derivative of $v$ along
  $\gamma$ is the orthogonal projection $\nabla_{\gamma}v$ of
  $dv/dt$ onto the thangent plane $T_{\gamma(t)} S$ at a point $\gamma(t)$.
\end{defn}

\begin{defn}
  $v$ is said to be parallel along $\gamma$ if
  $\nabla_{\gamma}v = 0$ at every point of $\gamma$.
\end{defn}

\begin{proposition}
  A tangent vector field $v$ is parallel along a curve $\gamma$ on a
  surface $S$ if and only if $\dot{v}$ is perpendicular to the
  tangent plane of $S$ at all points of $\gamma$.
\end{proposition}

\begin{proposition}[Gauss Equations]
  Let $\sigma(u, v)$ be a surface patch with first and second fundamental
  forms
  $E du^2 + 2 F du dv + G dv^2$ and
  $L du^2 + 2 M du dv + N dv^2$.
  Then
  \begin{align*}
    \sigma_uu &= \Gamma_{11}^1 \sigma_u + \Gamma_{11}^2 \sigma_v + L \bm{N},\\
    \sigma_uv &= \Gamma_{12}^1 \sigma_u + \Gamma_{12}^2 \sigma_v + M \bm{N},\\
    \sigma_vv &= \Gamma_{22}^1 \sigma_u + \Gamma_{22}^2 \sigma_v + N \bm{N},\\
  \end{align*}
  where
  \begin{gather*}
    \Gamma_{11}^1 = \frac{GE_u - 2 F F_u + F E_v}{2(EG - F^2)}, \, \Gamma_{11}^2 = \frac{2EF_u - EE_v -FE_u}{2(EG - F^2)},\\
    \Gamma_{12}^1 = \frac{GE_v - FG_u}{2(EG - F^2)}, \, \Gamma_{12}^2 = \frac{EG_u - FE_v}{2(EG - F^2)},\\
    \Gamma_{22}^1 = \frac{2GF_v - GG_u - FG_v}{2(EG - F^2)}, \, \Gamma_{22}^2 = \frac{EG_v - 2FF_v + FG_u}{2(EG - F^2)}.
  \end{gather*}
  The six $\Gamma$ coefficients in these formulas are called Christoffel symbols.
\end{proposition}

\begin{note}
  Christoffel symbols depend only on the first fundamental form of $\sigma$.
\end{note}

\begin{proposition}
  Let $\gamma(t) = \sigma(u(t), v(t))$ be a curve on a surface patch
  $\sigma$, and let $\bm{v}(t) = \alpha(t) \sigma_u + \beta(t) \sigma_v$
  be a tangent vector field along $\gamma$, where $\alpha$ and $\beta$
  are smooth functions of $t$. Then $\bm{v}$ is parallel along
  $\gamma$ if and only if the following equations are satisfied
  \begin{align*}
    \dot{\alpha} + (\Gamma_{11}^1 \dot{u} + \Gamma_{12}^1)\alpha + (\Gamma_{12}^1 \dot{u} + \Gamma_{22}^1 \dot{v})\beta = 0\\
    \dot{\beta} + (\Gamma_{11}^2 \dot{u} + \Gamma_{12}^2)\alpha + (\Gamma_{12}^2 \dot{u} + \Gamma_{22}^2 \dot{v})\beta = 0.
  \end{align*}
\end{proposition}

\begin{corollary}
  Let $\gamma$ be a curve on a surface $S$ and let $v_0$ be a tangent
  vector of $S$ at the point $\gamma(t_0)$.
  Then, there is exactly one tangent vector field $v$ that is parallel
  along $\gamma$ and is such that $v(t_0) = v_0$.
\end{corollary}

If $p$ and $q$ are two points on a curve $\gamma$ on a surface $S$,
the covariant derivative enables us to associate to any vector in the
tangent plane $T_p S$ a vector in the tangent plane $T_q S$.

\begin{defn}
  With the above notation, the map
  $\Pi_{\gamma}^{pq}: T_p S \to T_q S$ that takes
  $v_0 \in T_p S$ to $v_1 \in T_q S$ is called
  parallel transport from $p$ to $q$ along $\gamma$.
\end{defn}

\begin{proposition}
  With the above notation,
  \begin{itemize}
    \item $\Pi_{\gamma}^{pq}: T_p S \to T_q S$ is a linear map.
    \item $\Pi_{\gamma}^{pq}$ is an isometry.
  \end{itemize}
\end{proposition}