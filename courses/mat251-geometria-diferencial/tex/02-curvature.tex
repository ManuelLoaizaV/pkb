\section{How much does a curve curve?}

\subsection{Curvature}

\begin{defn}
  If $\gamma$ is a unit-speed curve with parameter $t$, its curvature $\kappa(t)$
  at the point $\gamma(t)$ is defined to be $\|\ddot{\gamma}(t)\|$.
\end{defn}

\begin{proposition}
  Let $\gamma(t)$ be a regular curve in $\BR^3$. Then its curvature is
  \[
    \kappa = \frac{\|\ddot{\gamma} \times \dot{\gamma}\|}{\|\dot{\gamma}\|^3},
  \]
  where the $\times$ indicates the cross product.
\end{proposition}

\subsection{Plane curves}

Suppose that $\gamma(t)$ is a unit-speed vector curve in $\BR^2$. Let
\[
  t = \dot{\gamma}
\]
be the tangent vector of $\gamma$, note that $t$ is a unit vector.
There are two unit vectors perpendicular to $t$; we make a choice by defining $n_s$,
the signed unit normal of $\gamma$, to be the unit vector obtained by rotating $t$
anticlockwise by $\pi/2$.
$\dot{t} = \ddot{\gamma}$ is perperndicualr to $t$, and hence parallel to $n_s$.
Thus, there is a scalar $\kappa_s$ such that
\[
  \ddot{\gamma} = \kappa_s n_s.
\]
$\kappa_s$ is called the signed curvature of $\gamma$. Since $\|n_s\| = 1$, we have
\[
  \kappa = \|\ddot{\gamma}\| = \|\kappa_s n_s\| = |\kappa_s|,
\]
so the curvature of $\gamma$ is the absolute value of its signed curvature.

% TODO: Add theorems related to turning angle

\begin{corollary}
  The total signed curvature of a closed plane curve is an integer multiple of $2\pi$.
\end{corollary}

\begin{theorem}
  Let $k: (\alpha, \beta) \to \BR$ be any smooth function.
  Then, there is a unit-speed curve
  $\gamma: (\alpha, \beta) \to \BR^2$ whose signed curvature is $k$.
  Further, if $\tilde \gamma: (\alpha, \beta) \to \BR^2$ is any other unit-speed curve
  whose signed curvature is $k$, there is a direct isometry $M$ of $\BR^2$ such that
  \[
    \tilde \gamma(s) = M(\gamma(s)) \text{ for all } s \in (\alpha, \beta).
  \]
\end{theorem}

\subsection{Space curves}

Let $\gamma(s)$ be a unit-speed curve in $\BR^3$, and let
$t = \dot{\gamma}$ be its unit tangent vector.
If the curvature $\kappa(s)$ is non-zero, we define the principal normal of $\gamma$
at the point $\gamma(s)$ to be the vector
\[
  n(s) = \frac{1}{\kappa(s)} \dot{t}(s).
\]
Since $\|\dot{t}\| = \kappa$, $n$ is a unit vector.
Further, $t \cdot \dot{t} = 0$ so $t$ and $n$ are actually perpendicular unit vectors.
It follows that
\[
  b = t \times n  
\]
is a unit vector perpendicular to both $t$ and $n$.
The vector $b(s)$ is called the binormal vector of $\gamma$ at the point $\gamma(s)$.
This, $\{t, n, b\}$ is an orthonormal basis of $\BR^3$ and is right-handed
\[
  b = t \times n, n = b \times t, t = n \times b.  
\]
Being perpendicular to both $t$ and $b$, $\dot{b}$ must be parallel to $n$, so
\[
  \dot{b} = -\tau n,
\]
for some scalar $\tau$, which is called the torsion of $\gamma$.

\begin{proposition}
  Let $\gamma(t)$ be a regular curve in $\BR^3$ with nowhere-vanishing curvature.
  Then, its torsion is given by
  \[
    \tau = \frac{
      (\dot{\gamma} \times \ddot{\gamma}) \cdot \dddot{\gamma}
    }{
      \|\dot{\gamma} \times \ddot{\gamma}\|^2
    }.
  \]
\end{proposition}


\begin{proposition}
  Let $\gamma$ be a regular curve in $\BR^3$ with nowhere-vanishing curvature.
  Then, the imagen of $\gamma$ is contained in a plane if and only if $\tau$ is zero at every point of the curve.
\end{proposition}

\begin{theorem}
  Let $\gamma$ be a unit-speed curve in $\BR^3$ with nowhere vanishing curvature. Then
  \begin{align*}
    \dot{t} &= \kappa n \\
    \dot{n} &= -\kappa t + \tau b \\
    \dot{b} &= - \tau n.
  \end{align*}
  These equations are called the Frenet-Serret equations.
\end{theorem}

\begin{note}
  The matrix
  \[
    \dot{
      \begin{pmatrix}
        t\\
        n\\
        b
      \end{pmatrix}
    }
    =
    \begin{pmatrix}
      0 & \kappa & 0 \\
      -\kappa & 0 & \tau \\
      0 & -\tau & 0
    \end{pmatrix}
    \begin{pmatrix}
      t\\
      n\\
      b
    \end{pmatrix}
  \]
  which expresses $\dot{\tau}, \dot{n}, \dot{b}$ in terms of $t, n$ and $b$ is skew-symmetric.
\end{note}

\begin{proposition}
  Let $\gamma$ be a unit-speed curve in $\BR^3$ with constant curvature and zero torsion.
  Then, $\gamma$ is a parametrization of (part of) a circle.
\end{proposition}

\begin{theorem}
  Let $\gamma(s)$ and $\tilde \gamma(s)$ be two unit-speed curves in $\BR^3$ with the same curvature
  $\kappa(s) > 0$ and the same torsion $\tau(s)$ for all $s$.
  Then, there is a direct isometry $M$ of $\BR^3$ such that
  \[
    \tilde \tau(s) = M(\gamma(s)) \text{ for all } s.
  \]
  Further, if $k$ and $t$ are smooth functions with $k > 0$ everywhere,
  there is a unit-speed curve in $\BR^3$ whose curvature is $k$ and whose torsion is $t$.
\end{theorem}