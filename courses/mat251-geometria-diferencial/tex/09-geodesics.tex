\section{Geodesics}

\subsection{Definition and basic properties}

\begin{defn}
  A curve $\gamma$ on a surface $S$ is called a geodesic if $\ddot{\gamma}(t)$ is zero
  or perpendicular to the tangent plane of the surface at the point $\gamma(t)$, i.e.,
  parallel to its unit noirmal, for all values of the parameter $t$.
\end{defn}

\begin{remark}
  Equivalently, $\gamma$ is geodesic if and only if its tangent vector $\dot{\gamma}$
  is parallel along $\gamma$.
\end{remark}

\begin{proposition}
  Any geodesic has constant speed.
\end{proposition}

\begin{proposition}
  A unit-speed curve on a surface is a geodesic if and only if its geodesic
  curvature is zero everywhere.
\end{proposition}

\begin{proposition}
  Any (part of a) straight line on a surface is a geodesic.
\end{proposition}

\begin{proposition}
  Any normal section of a surface is a geodesic.
\end{proposition}

\subsection{Geodesic equations}

\begin{theorem}
  A curve $\gamma$ on a surface $S$ is a geodesic if and only if,
  for any part $\gamma(t) = \sigma(u(t), v(t))$ of $\gamma$
  contained in a surface patch $\sigma$ of $S$, the following
  two equations are satisfied:
  \begin{align*}
    \frac{d}{dt}(E\dot{u} + F\dot{v}) & = \frac{1}{2}(E_u\dot{u}^2 + 2F_u\dot{u}\dot{v} + G_u\dot{v}^2),\\
    \frac{d}{dt}(E\dot{u} + F\dot{v}) &= \frac{1}{2}(E_v\dot{u}^2 + 2F_v\dot{u}\dot{v} + G_v\dot{v}^2).
  \end{align*}
  These differential equations are called the geodesic equations.
\end{theorem}

\begin{proposition}
  A curve $\gamma$ of a surface $S$ is a geodesic if and only if,
  for any part $\gamma(t) = \sigma(u(t), v(t))$ of $\gamma$ contained
  in a surface patch $\gamma$ of $S$, the following two equations are satisfied:
  \begin{align*}
    \ddot{u} + \Gamma_{11}^1 \dot{u}^2 + 2 \Gamma_{12}^1 \dot{u}\dot{v} + \Gamma_{22}^1 \dot{v}^2 &= 0\\
    \ddot{v} + \Gamma_{11}^2 \dot{u}^2 + 2 \Gamma_{12}^2 \dot{u}\dot{v} + \Gamma_{22}^2 \dot{v}^2 &= 0\\
  \end{align*}
\end{proposition}

\begin{proposition}
  Let $p$ be a point of a surface $S$, and let $t$ be a unit tangent vector
  to $S$ at $p$. Then, there exist a unique unit-speed geodesic $\gamma$ on $S$ which
  passes through $p$ and has tangent vector $t$ there.
\end{proposition}

In short, there is a unique geodesic through any given point of a surface in any given tangent direction.

\begin{corollary}
  Any local isometry between two surfaces takes the geodesics of one surface to the geodesics of the other.
\end{corollary}