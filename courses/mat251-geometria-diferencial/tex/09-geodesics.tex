\section{Geodesics}

\subsection{Definition and basic properties}

\begin{defn}
  A curve $\gamma$ on a surface $S$ is called a geodesic if $\ddot{\gamma}(t)$ is zero
  or perpendicular to the tangent plane of the surface at the point $\gamma(t)$, i.e.,
  parallel to its unit noirmal, for all values of the parameter $t$.
\end{defn}

\begin{remark}
  Equivalently, $\gamma$ is geodesic if and only if its tangent vector $\dot{\gamma}$
  is parallel along $\gamma$.
\end{remark}

\begin{proposition}
  Any geodesic has constant speed.
\end{proposition}

\begin{proposition}
  A unit-speed curve on a surface is a geodesic if and only if its geodesic
  curvature is zero everywhere.
\end{proposition}

\begin{proposition}
  Any (part of a) straight line on a surface is a geodesic.
\end{proposition}

\begin{proposition}
  Any normal section of a surface is a geodesic.
\end{proposition}